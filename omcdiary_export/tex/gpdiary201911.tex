% Options for packages loaded elsewhere
\PassOptionsToPackage{unicode}{hyperref}
\PassOptionsToPackage{hyphens}{url}
%
\documentclass[
]{article}
\usepackage{lmodern}
\usepackage{amssymb,amsmath}
\usepackage{ifxetex,ifluatex}
\ifnum 0\ifxetex 1\fi\ifluatex 1\fi=0 % if pdftex
  \usepackage[T1]{fontenc}
  \usepackage[utf8]{inputenc}
  \usepackage{textcomp} % provide euro and other symbols
\else % if luatex or xetex
  \usepackage{unicode-math}
  \defaultfontfeatures{Scale=MatchLowercase}
  \defaultfontfeatures[\rmfamily]{Ligatures=TeX,Scale=1}
\fi
% Use upquote if available, for straight quotes in verbatim environments
\IfFileExists{upquote.sty}{\usepackage{upquote}}{}
\IfFileExists{microtype.sty}{% use microtype if available
  \usepackage[]{microtype}
  \UseMicrotypeSet[protrusion]{basicmath} % disable protrusion for tt fonts
}{}
\makeatletter
\@ifundefined{KOMAClassName}{% if non-KOMA class
  \IfFileExists{parskip.sty}{%
    \usepackage{parskip}
  }{% else
    \setlength{\parindent}{0pt}
    \setlength{\parskip}{6pt plus 2pt minus 1pt}}
}{% if KOMA class
  \KOMAoptions{parskip=half}}
\makeatother
\usepackage{xcolor}
\IfFileExists{xurl.sty}{\usepackage{xurl}}{} % add URL line breaks if available
\IfFileExists{bookmark.sty}{\usepackage{bookmark}}{\usepackage{hyperref}}
\hypersetup{
  hidelinks,
  pdfcreator={LaTeX via pandoc}}
\urlstyle{same} % disable monospaced font for URLs
\setlength{\emergencystretch}{3em} % prevent overfull lines
\providecommand{\tightlist}{%
  \setlength{\itemsep}{0pt}\setlength{\parskip}{0pt}}
\setcounter{secnumdepth}{-\maxdimen} % remove section numbering

\date{}

\begin{document}

\hypertarget{ux84bcux767dux7403ux65e5ux8a8c0034-20191101}{%
\section{蒼白球日誌0034
2019/11/01}\label{ux84bcux767dux7403ux65e5ux8a8c0034-20191101}}

\hypertarget{ux65e5ux671f-date}{%
\subsection{日期 Date}\label{ux65e5ux671f-date}}

\begin{itemize}
\tightlist
\item
  世界協調時間2019年(中華民國108年,令和1年)11月1日 / Unix 紀元 18201 日
  / 星期五 / 蒼白球紀元第34日
\item
  November 01, 2019 (UTC) / 18201 days since Unix Epoch / Friday /
  Globus Pallidum day 34
\item
  特殊註記:
\end{itemize}

\hypertarget{ux5e74ux9f61-age}{%
\subsection{年齡 Age}\label{ux5e74ux9f61-age}}

\begin{itemize}
\tightlist
\item
  33 years 6 months 9 days old / 2 years 0 months 20 days after
  acquiring ROC Surgical Pathology Licence
\item
  33 歲 6 個月 9 天 / 成為病理專科醫師 2 年 0 個月 20 天
\end{itemize}

\hypertarget{ux672cux6587-content}{%
\subsection{本文 Content}\label{ux672cux6587-content}}

\begin{enumerate}
\def\labelenumi{\arabic{enumi}.}
\item
  今天把吃的放在正式條目吧,別放雜記

  因為早上聽的課實在太無聊了,老師講他自己的專業項目,我全無興趣,只能滑手機度過。這個狀況下不如來寫寫中午路過莒光新城{[}1{]}時吃的麵店「大城小吃」。

  莒光新城不是普通眷村,是與一中商圈{[}3{]}鄰接的住宅區,直接面對一中街的挑戰,開在這種地段還能夠常常門庭若市,想必一定好吃。這個想法,再加上對於「眷村」的刻板印象,讓我原本很想嚐嚐他麵食有多高明。無奈的是,因為肚子真的很餓,下午晚上又要趕片子跟團練,吃麵可能會撐不過去,只好在麵店點了爌肉飯來吃。

  麵店的爌肉飯,聽起來就不期不待,沒想到一吃簡直是中到大獎,這個爌肉飯水準之高,可能已經超過了第二市場名店的水準。充滿光澤的三層肉滷成漂亮的紅磚色,咬下去居然酥爛到入口即化,放出美麗的醬油香{[}4{]}以及多種中藥香料的味道,正在納悶他加的香料有哪些的時候,口中咬到了一片月桂葉,喔!原來是這個啊!真是太棒了!

  所附的酸筍跟油豆腐也頗具水準,與爌肉及白飯搭配成雋永的四重奏,是布拉姆斯嗎,或是舒伯特?還是蕭士塔高維契?總之這爌肉四重奏是大師之作。而另外點的粉腸湯更是絕妙得無法形容,來麵店吃飯還真的吃對了。爌肉飯50元{[}5{]},粉腸湯35元。「台灣第一味」甘蔗青茶50元(「先生不好意思,雖然你出示了識別證,但是兩點以後才有打折喔!」)。

  晚餐因為團練,所以C院請客。
\item
  團練

  重頭戲是學弟妹跟母后的鋼琴五重奏,因為母后之前嫌那首曲子的結尾編得太冷場太難聽了,於是我就硬插了一個無比激情的曲子進去收尾,編的時候把我自己操到作息亂掉(見蒼白球日誌0030-0033),然後今天練的時候把他們操到氣喘吁吁。這幾個人應該很久沒有拉這麼難的曲子了。

  不過痛苦是有代價的,這樣合起來效果真的還不錯,再配上一個實力高超的鈴鼓手(醫事室主任的手下,所以被逼著跟我們演出),相信應該是11月11日醫師節的亮點了。
\item
  馬橋詞典閱讀筆記

  馬橋詞典的結語,跟蒼白球日誌想要表達的核心很像,茲節錄了一段如下:
  「從嚴格的意義上來說,所謂``共同的語言'',永遠是人類一個遙遠的目標。如果我們下希望交流成為一種互相抵銷,互相磨滅,我們就必須對交流保持警覺和抗拒,在妥協中守護自己某種頑強的表達------這正是一種良性交流的前提。這就意味著,人們在說話的時候,如果可能的話,每個人都需要一本自己特有的詞典。」
\end{enumerate}

\hypertarget{ux6ce8ux91cb-comment}{%
\subsection{注釋 Comment}\label{ux6ce8ux91cb-comment}}

{[}1{]}
為一眷村{[}2{]}改建的大型集合住宅,居住者主要以空軍將官為主,因此算是頗為高級的眷村,其飲食與器物也比一般眷村要有格調。

{[}2{]}
眷村是指台灣自1949年起至1960年代,來自中國大陸各省的中華民國國軍及其眷屬,因第二次國共內戰失利而隨中華民國政府遷徙至台灣後,政府機關為其興建或者配置的村落。

{[}3{]} 有關大型商圈「一中街」,請見蒼白球日誌0013。

{[}4{]}
我是濁水溪出身的人,濁水溪產最棒的醬油,所以用好醬油我一吃就知道。

{[}5{]} 新台幣計價。有關新台幣請參見蒼白球日誌0007。

\hypertarget{ux9644ux9304-appendix}{%
\subsection{附錄 Appendix}\label{ux9644ux9304-appendix}}

\hypertarget{ux84bcux767dux7403ux65e5ux8a8c0035-20191102}{%
\section{蒼白球日誌0035
2019/11/02}\label{ux84bcux767dux7403ux65e5ux8a8c0035-20191102}}

\hypertarget{ux65e5ux671f-date-1}{%
\subsection{日期 Date}\label{ux65e5ux671f-date-1}}

\begin{itemize}
\tightlist
\item
  世界協調時間2019年(中華民國108年,令和1年)11月2日 / Unix 紀元 18202 日
  / 星期六 / 蒼白球紀元第35日
\item
  November 02, 2019 (UTC) / 18202 days since Unix Epoch / Saturday /
  Globus Pallidum day 35
\item
  特殊註記:
\end{itemize}

\hypertarget{ux5e74ux9f61-age-1}{%
\subsection{年齡 Age}\label{ux5e74ux9f61-age-1}}

\begin{itemize}
\tightlist
\item
  33 years 6 months 10 days old / 2 years 0 months 21 days after
  acquiring ROC Surgical Pathology Licence
\item
  33 歲 6 個月 10 天 / 成為病理專科醫師 2 年 0 個月 21 天
\end{itemize}

\hypertarget{ux672cux6587-content-1}{%
\subsection{本文 Content}\label{ux672cux6587-content-1}}

\begin{enumerate}
\def\labelenumi{\arabic{enumi}.}
\item
  很慚愧

  今天遠赴P市講了一場很爛的演講以後,去男友店裡坐坐,發現他忙進忙出招呼客人,忙到沒時間寫日記。看到他忙碌的情形突然覺得很慚愧,我每天還可以空時間出來寫蒼白球日誌,應該是因為很多工作偷懶擺爛的緣故。

  想到這裡就決定要來充實一下,坐車回C市以後馬上到辦公室來進行病理工作,並且拿研究用的影像回家稍微閱讀一下。雖然這個努力工作可能是曇花一現,明天搞不好就繼續擺爛了,但是稍微可以彌補一下我短期的慚愧感吧。
\item
  續第一點,之男友店裡的情形

  這家店真的很會做生意,在週六下午這個時段一直維持坐滿八成的榮景,但最讓我詫異的倒不是男友經營事業如此長袖善舞,是人們居然對帶點刺激酸味的東西趨之若鶩,甘香微苦的美好品項反而乏人問津,北部人真的讓我難以理解。

  我們濁水溪人講究「甘甘香香軟軟」,酸不可以過酸,苦不可以過苦,鹹不可以過鹹,辣不可以過辣,所以我對於甜食的想像都是砂糖與香料,甜滋滋香噴噴軟綿綿的東西,例如說什麼呢,嗯,雖然香蕉飴真的很濁水,但是不要提香蕉飴好了,例如說泡芙,薄荷巧克力冰淇淋,奶油楓糖煎香蕉,或者是用大量砂糖丁香跟肉桂粉煮到酸味全失的蘋果餡。北部人不吃甘甘軟軟這一套,反而一到店裡就吃酸,讓我突然發現濁水溪流域其實並不能代表台灣,我的眼界真的太狹窄了\ldots\ldots{}
\item
  雜記:物價與其他{[}1{]}

  高鐵P市到C市來回共1350元,P市中心一家看起來很破舊的自助餐店午餐100元(爌肉,地瓜葉,花菜,莧菜,不怎麼樣又貴卻生意很好,P市居大不易!),「Fruit
  Drink」青茶30元,「樂檸漢堡」培磨牛肉堡雞塊青茶共179元(沒吃過樂檸的人請務必吃至少一次,身為台灣人沒吃過樂檸是很可恥的)
\end{enumerate}

\hypertarget{ux6ce8ux91cb-comment-1}{%
\subsection{注釋 Comment}\label{ux6ce8ux91cb-comment-1}}

{[}1{]} 新台幣計價。有關新台幣請參見蒼白球日誌0007。

\hypertarget{ux9644ux9304-appendix-1}{%
\subsection{附錄 Appendix}\label{ux9644ux9304-appendix-1}}

\hypertarget{ux84bcux767dux7403ux65e5ux8a8c0036-20191103}{%
\section{蒼白球日誌0036
2019/11/03}\label{ux84bcux767dux7403ux65e5ux8a8c0036-20191103}}

\hypertarget{ux65e5ux671f-date-2}{%
\subsection{日期 Date}\label{ux65e5ux671f-date-2}}

\begin{itemize}
\tightlist
\item
  世界協調時間2019年(中華民國108年,令和1年)11月3日 / Unix 紀元 18203 日
  / 星期日 / 蒼白球紀元第36日
\item
  November 03, 2019 (UTC) / 18203 days since Unix Epoch / Sunday /
  Globus Pallidum day 36
\end{itemize}

\hypertarget{ux5e74ux9f61-age-2}{%
\subsection{年齡 Age}\label{ux5e74ux9f61-age-2}}

\begin{itemize}
\tightlist
\item
  33 years 6 months 11 days old / 2 years 0 months 22 days after
  acquiring ROC Surgical Pathology Licence
\item
  33 歲 6 個月 11 天 / 成為病理專科醫師 2 年 0 個月 22 天
\end{itemize}

\hypertarget{ux672cux6587-content-2}{%
\subsection{本文 Content}\label{ux672cux6587-content-2}}

\begin{enumerate}
\def\labelenumi{\arabic{enumi}.}
\item
  母后大人事蹟之一,感受到文化衝擊的醫師節

  前陣子Y縣醫師節餐會的請帖寄到家中,上面載明可以免費攜眷一人,若繳六百塊就可以攜眷兩人。當時我其實有暗示母后說不要去,無奈的是他聽到大飯店宴席就變得超興高采烈,我實在說不過他,只好被迫攜母后跟父皇參加。

  結果到了現場,母后果然後悔了,大大後悔了,他完全想不到那是一個比應酬還應酬,比官僚還官僚的飯局。因為破事實在太多,我的中文能力不夠把它整理成良好的段落,所以只好浪費空間條列如下。

  \begin{itemize}
  \item
    先花40分鐘頒發服務滿三十年醫師的獎狀,之後才准上菜。母后因此在旁邊的麵包店買了麵包墊肚子。
  \item
    各個醫院的高層輪流到各桌敬酒,所以母后跟父皇也被迫跟我一起每五分鐘起立一次。
  \item
    我的直屬長官剛好坐跟我同一桌,母后跟父皇只好跟我一起陪笑。
  \item
    某政府高層帶了他的一團屬下來一起站在台上一邊搖政府宣傳標語一邊唱卡拉OK。(唯一欣慰的事情是他的屬下裡面有很鮮很鮮的小鮮肉{[}1{]}。
  \item
    然後Y縣醫師公會所有理監事陪著高層一起唱卡拉OK,屬下繼續搖標語。
  \item
    大家很想喝的最後一道菜 -
    雞湯,為了要等立委來致詞因此延遲了半小時還無法上菜,大部分人,包含我們都忍不住離席了。
  \item
    母后回到C市後決定用鹽酥雞填飽肚子。
  \end{itemize}

  母后跟父皇一輩子都待在吃的文化圈,對他們來講無論平常怎麼打官腔,吃的時候至少要吃得開心,所以很難想像有這麼不開心的宴席。好,現在他體會到了,未來所有醫療圈宴席他應該都不會吵著要跟了。
\item
  母后大人事蹟之二,清唱劇

  這個清唱劇是母后大人為了我還有父皇大人苦心安排的曝光機會,原本應該好好做,不過我實在沒有足夠的編曲實力可以撐起這種大場,也太過懶惰,所以最後他只好安排現成的歌曲編一編,然後讓我出一些以前寫過的現成素材,勉強做一個拼裝車清唱劇。

  如果對我的作品量有理解的人,應該會知道我沒很多現成素材可以用(懶惰所以寫很少),不過從電腦裡面苦心搜尋以後,居然還真的有三四首可以用,都是過去做配樂的時候被老闆淘汰的東西。如果來聽這場清唱劇的那些大人物知道這是場廢品再利用的回收劇,不知會做何感想。
\item
  母后大人事蹟之三,評鑑資料

  母后大人去參加某個老人教育的評鑑,為了要討好那些委員,必須把明明是教音樂這件事情包裝成預防失智的偉大教育,所以我就被迫生了一堆腦科學的名詞讓他上去唬爛,今天早上還花了三小時教會他怎麼講得滿口都是似是而非半對半錯的偽腦科學。我其實很擔心,那些程度很差的老人教育科系大學教授,會被他唬爛到真的相信學音樂可以防失智,學音樂可以變聰明。

  雖然說其實我也覺得學音樂可以變聰明就是了\ldots\ldots 但是那到底是我的自我洗腦,還是證據力真的足夠,這可能還要多多釐清。
\end{enumerate}

\hypertarget{ux6ce8ux91cb-comment-2}{%
\subsection{注釋 Comment}\label{ux6ce8ux91cb-comment-2}}

{[}1{]} 此名詞請見蒼白球日誌0030。

\hypertarget{ux9644ux9304-appendix-2}{%
\subsection{附錄 Appendix}\label{ux9644ux9304-appendix-2}}

\hypertarget{ux84bcux767dux7403ux65e5ux8a8c0037-20191104}{%
\section{蒼白球日誌0037
2019/11/04}\label{ux84bcux767dux7403ux65e5ux8a8c0037-20191104}}

\hypertarget{ux65e5ux671f-date-3}{%
\subsection{日期 Date}\label{ux65e5ux671f-date-3}}

\begin{itemize}
\tightlist
\item
  世界協調時間2019年(中華民國108年,令和1年)11月4日 / Unix 紀元 18204 日
  / 星期一 / 蒼白球紀元第37日
\item
  November 04, 2019 (UTC) / 18204 days since Unix Epoch / Monday /
  Globus Pallidum day 37
\item
  特殊註記:
\end{itemize}

\hypertarget{ux5e74ux9f61-age-3}{%
\subsection{年齡 Age}\label{ux5e74ux9f61-age-3}}

\begin{itemize}
\tightlist
\item
  33 years 6 months 12 days old / 2 years 0 months 23 days after
  acquiring ROC Surgical Pathology Licence
\item
  33 歲 6 個月 12 天 / 成為病理專科醫師 2 年 0 個月 23 天
\end{itemize}

\hypertarget{ux672cux6587-content-3}{%
\subsection{本文 Content}\label{ux672cux6587-content-3}}

\begin{enumerate}
\def\labelenumi{\arabic{enumi}.}
\item
  蒼子(2007)(已絕版)

  最近因為常常思考資料存續的問題,因此稍微盤點了一下手上持有的老紙本資料,過程中意外發現了這本被我放在書架一角,從未仔細翻閱的書。

  蒼子是我大學學長,把自己教物理家教、當醫學生跟長期交不到女朋友的經歷寫成了一整本的碎碎念,而且還自費出版,分送給一些親友。而當時我就是其中一個被視為親友的人,交到我手上的時候書的扉頁還蓋了他的實習醫學生職章,超級宇宙無敵中二{[}1{]}的。

  坦白講此人駕馭文法的能力算及格,但因為書中內容實在是太過碎碎念,所以以散文來講實在不太好看,當時會留著這本書,恐怕只是因為我暗戀了蒼子一小段時間{[}2{]}。然而,十二年後對於蒼子的感情早就退去,這本書卻因為其不好看的特質,意外地感覺重要了起來。畢竟,在這個社群網路的時代,醫學生跟家教老師的生活紀錄各種裝模作樣假惺惺,像這本書這種樸拙真實的紀錄,已經變得非常稀有了。而且,還紀錄了一個智慧型手機不普及的年代C市的面貌。而且,還是作者蓋章的絕版書。

  所以我老是鼓勵男友多寫一些甜點星球物語的事情,畢竟在世界協調時間2019年(中華民國108年,令和1年)看起來像碎碎念的東西,也許在2029年就是珍貴實錄了。
\item
  雜記:物價與其他{[}3{]}

  \begin{itemize}
  \tightlist
  \item
    認真練了一整個晚上的小提琴。好痛苦。好討厭拉琴。好後悔醫師節說要上台拉琴。
  \item
    然後為了練琴,臉皮很厚地把所有外院的報告都拖到超過期限,因為外院報告不會罰錢,也不算在考績裡面。真是個不老實工作的病理人。
  \item
    修小提琴弓2000元,買指揮棒與松香1000元,C院午餐72元(紫米飯,芥蘭,洋蔥豬柳,白菜,青花菜),「清心福全」普洱茶5元(使用line
    point20點{[}4{]})
  \end{itemize}
\end{enumerate}

\hypertarget{ux6ce8ux91cb-comment-3}{%
\subsection{注釋 Comment}\label{ux6ce8ux91cb-comment-3}}

{[}1{]}
中二病(日語:中二病、厨二病),簡稱中二,是源自日本的網路流行語,泛指一種自我認知心態,用以形容一些經常自以為是地活在自己世界或做出自我滿足的特別言行的人,是青春期特有的價值觀的總稱。雖然稱「病」,但和醫學上的「疾病」沒有任何關係。中文中接近中二病意象的字詞有年少輕狂、少不更事等,但都未能精準地描述中二病。

{[}2{]}
這個地方應該要註釋什麼叫做異男忘,但這名詞對我來說實在太過酸澀,不想解釋,請看不懂的人自行查詢別的來源。

{[}3{]} 新台幣計價。有關新台幣請參見蒼白球日誌0007。

{[}4{]} 某些信用卡消費後給予回饋的點數,可以在使用line
pay這種行動支付工具消費時視為現金抵用。

\hypertarget{ux9644ux9304-appendix-3}{%
\subsection{附錄 Appendix}\label{ux9644ux9304-appendix-3}}

\hypertarget{ux84bcux767dux7403ux65e5ux8a8c0038-20191105}{%
\section{蒼白球日誌0038
2019/11/05}\label{ux84bcux767dux7403ux65e5ux8a8c0038-20191105}}

\hypertarget{ux65e5ux671f-date-4}{%
\subsection{日期 Date}\label{ux65e5ux671f-date-4}}

\begin{itemize}
\tightlist
\item
  世界協調時間2019年(中華民國108年,令和1年)11月5日 / Unix 紀元 18205 日
  / 星期二 / 蒼白球紀元第38日
\item
  November 05, 2019 (UTC) / 18205 days since Unix Epoch / Tuesday /
  Globus Pallidum day 38
\end{itemize}

\hypertarget{ux5e74ux9f61-age-4}{%
\subsection{年齡 Age}\label{ux5e74ux9f61-age-4}}

\begin{itemize}
\tightlist
\item
  33 years 6 months 13 days old / 2 years 0 months 24 days after
  acquiring ROC Surgical Pathology Licence
\item
  33 歲 6 個月 13 天 / 成為病理專科醫師 2 年 0 個月 24 天
\end{itemize}

\hypertarget{ux672cux6587-content-4}{%
\subsection{本文 Content}\label{ux672cux6587-content-4}}

\begin{enumerate}
\def\labelenumi{\arabic{enumi}.}
\item
  又欠了男友還有diss中一點

  今天因為各種壓力的關係有點發病,對著男友還有diss中講一堆有的沒有的,倒各種情緒垃圾,回頭想想,覺得又欠了一點人情債。

  但反正這債我是永遠還不起的,從他們收留我流浪中的靈魂開始就欠下巨額了吧。
\item
  又把片子放著不寫了

  因為要提早衝回家剪下個禮拜一要表演的影片,然後練琴。說到我之所以今天可以準時練琴,也算是欠了人情債。表演前我決定要把琴弓修一修,樂器行原本無法一天給我,結果拜託了認識的交響樂團耆老,才硬逼著把弓提前修好\ldots\ldots{}
\item
  文法文法文法

  今天狀況實在太不好,且趕著要作影片,顧不得文法了,所以大家就原諒我前兩段寫得亂七八糟吧。
\end{enumerate}

\hypertarget{ux6ce8ux91cb-comment-4}{%
\subsection{注釋 Comment}\label{ux6ce8ux91cb-comment-4}}

狀況不好無法註釋

\hypertarget{ux9644ux9304-appendix-4}{%
\subsection{附錄 Appendix}\label{ux9644ux9304-appendix-4}}

\hypertarget{ux84bcux767dux7403ux65e5ux8a8c0039-20191106}{%
\section{蒼白球日誌0039
2019/11/06}\label{ux84bcux767dux7403ux65e5ux8a8c0039-20191106}}

\hypertarget{ux65e5ux671f-date-5}{%
\subsection{日期 Date}\label{ux65e5ux671f-date-5}}

\begin{itemize}
\tightlist
\item
  世界協調時間2019年(中華民國108年,令和1年)11月6日 / Unix 紀元 18206 日
  / 星期三 / 蒼白球紀元第39日
\item
  November 06, 2019 (UTC) / 18206 days since Unix Epoch / Wednesday /
  Globus Pallidum day 39
\item
  特殊註記:
\end{itemize}

\hypertarget{ux5e74ux9f61-age-5}{%
\subsection{年齡 Age}\label{ux5e74ux9f61-age-5}}

\begin{itemize}
\tightlist
\item
  33 years 6 months 14 days old / 2 years 0 months 25 days after
  acquiring ROC Surgical Pathology Licence
\item
  33 歲 6 個月 14 天 / 成為病理專科醫師 2 年 0 個月 25 天
\end{itemize}

\hypertarget{ux672cux6587-content-5}{%
\subsection{本文 Content}\label{ux672cux6587-content-5}}

\begin{enumerate}
\def\labelenumi{\arabic{enumi}.}
\item
  決定把日誌檔案打包每月一個檔案{[}1{]}

  多個小檔案視覺上非常混亂,而且在儲存媒體上面的讀寫性也不好,因此我決定每月打包一次日誌,把當月所有日期的日誌(已經都是用Markdown{[}2{]}寫的)先用程式合併成一個大的Markdown檔案,再輸出成html、pdf與純文字,這三種我覺得可以存續很久的格式。目前除了pdf輸出還沒完成以外,其他程式都已經寫完了,只要python還穩定存在的一天,每月打包應該都很輕鬆才對。
\item
  疲累

  做了一個晚上的影片剪輯之後,整個人的靈魂完全被榨乾了,不只寫片片的時候不專心,連上健身教練課都打瞌睡。影片剪輯真是太可怕了!以後這種工作我要花錢找人做。
\item
  食記{[}3{]}

  在C院跟健身房之間,有一家裝潢很粗糙,生意看起來很差的冰店,他的外觀看起來糟糕到,十年來我經過這家店數百次,沒有一次想要進去吃。

  可是今天除外。因為整個身心的狀態很類似宿醉,很想隨便補個甜食來刺激腦部,所以就進去吃了。「五種冰」55元,內含芋園、綠豆、紅豆、仙草、薏仁,所有的料都煮得非常的軟爛,而且相當甜膩,果然是家並不會有好口碑的店。但是,但是,以2019年的中華民國,55元的冰來講,他給我的分量可真有誠意啊,所有料都是一坨一坨的加的,尤其是該店在網路上評價最差的紅豆給得最慷慨,感覺好像在說:「紅豆賣不出去就一口氣倒給你吃吧,可憐人」。

  而且一口氣倒給我吃的,還不只紅豆。在老闆把冰端給我的時候,還同時附上了第二碗東西,同時輕柔地說「這個就請你吃了」。

  是一碗湯圓加布丁豆花。我看到這個畫面其實心裡面狂笑不止,想說「會把湯圓連著上面沾著的白色糯米黏液,一起澆在滿滿洋菜質感的布丁豆花上面讓他看起來像餿水的店,實在難怪生意不好」,但是想到老闆應該是因為我長得一副魯蛇臉加可憐小動物表情,才好心餵我吃這麼多東西,就決定要賣力地把滿滿的料還有詭異的湯圓布丁豆花吃到一點都不剩。超級滿足,好久沒有吃甜食吃到肚子脹了。魯蛇臉有時也是很好用的。

  滿肚子糖以後去一中街補一些鹽分稀釋。雖然滿街都是滷味,但不是我想要的鹽分,炸菇才是。小份綜合菇45,菇跟炸粉的味道可以,但鹽加得有點過多。無所謂了,反正我是來稀釋魯蛇湯圓的。

  最後以「花茶大師」桑菊薄荷茶50元作結。平常喝很解膩,但是在大甜大鹹之後喝桑菊居然有點噁心感,可能只有茶葉才敵得過重口味吧。
\end{enumerate}

\hypertarget{ux6ce8ux91cb-comment-5}{%
\subsection{注釋 Comment}\label{ux6ce8ux91cb-comment-5}}

{[}1{]} 指的是電腦檔案。有關這個時代的電腦請見蒼白球日誌0008,
0010及0011。

{[}2{]} 此檔案格式見蒼白球日誌0014。

{[}3{]} 新台幣計價。有關新台幣請參見蒼白球日誌0007。

\hypertarget{ux9644ux9304-appendix-5}{%
\subsection{附錄 Appendix}\label{ux9644ux9304-appendix-5}}

\hypertarget{ux84bcux767dux7403ux65e5ux8a8c0040-20191107}{%
\section{蒼白球日誌0040
2019/11/07}\label{ux84bcux767dux7403ux65e5ux8a8c0040-20191107}}

\hypertarget{ux65e5ux671f-date-6}{%
\subsection{日期 Date}\label{ux65e5ux671f-date-6}}

\begin{itemize}
\tightlist
\item
  世界協調時間2019年(中華民國108年,令和1年)11月7日 / Unix 紀元 18207 日
  / 星期四 / 蒼白球紀元第40日
\item
  November 07, 2019 (UTC) / 18207 days since Unix Epoch / Thursday /
  Globus Pallidum day 40
\item
  特殊註記:
\end{itemize}

\hypertarget{ux5e74ux9f61-age-6}{%
\subsection{年齡 Age}\label{ux5e74ux9f61-age-6}}

\begin{itemize}
\tightlist
\item
  33 years 6 months 15 days old / 2 years 0 months 26 days after
  acquiring ROC Surgical Pathology Licence
\item
  33 歲 6 個月 15 天 / 成為病理專科醫師 2 年 0 個月 26 天
\end{itemize}

\hypertarget{ux672cux6587-content-6}{%
\subsection{本文 Content}\label{ux672cux6587-content-6}}

\begin{enumerate}
\def\labelenumi{\arabic{enumi}.}
\item
  研究重啟的第一天

  兩場演出的籌備稍微告一段落以後,今天終於可以開始做一些研究的事情。雖然說得好像很正經在做什麼偉大的事業,但內容其實很無聊,就是打開電腦,標示一點正常腦部的切片影像,給興大的研究生跑程式而已。

  有關機器學習的研究,在病理醫師這一端的工作就是這麼枯燥乏味,精采程度不如軟體工程方面的演算法,更不如硬體方面的機械設計。雖然這樣說,但病理醫師的資料輸入依然是整個研究不可或缺的一部分,畢竟這不像一般的圖像辨識,可以雇用普通大眾,病理切片只能由病理醫師判讀。

  醫師軟體硬體金三角,三足鼎立,缺一不可,身為三足其中之一,或許我也該有點榮譽感,不該覺得自己都在打雜。
\item
  練小提琴

  換過弓毛,然後讓大師開光過的琴弓拉起來有如神助,覺得自己的琴藝瞬間進步了,下周一可以好好去表演了。這個月該請大師吃頓好料才對。
\item
  增肥

  發現自己減肥成功以後,連續大吃大喝了好幾周,結果今天照鏡子發現臉變圓了,而且痘痘復發了。果然要復胖是很容易的呢,下周開始吃低卡餐跟跑步。
\end{enumerate}

\hypertarget{ux6ce8ux91cb-comment-6}{%
\subsection{注釋 Comment}\label{ux6ce8ux91cb-comment-6}}

懶得注

\hypertarget{ux9644ux9304-appendix-6}{%
\subsection{附錄 Appendix}\label{ux9644ux9304-appendix-6}}

\hypertarget{ux84bcux767dux7403ux65e5ux8a8c0041-20191108}{%
\section{蒼白球日誌0041
2019/11/08}\label{ux84bcux767dux7403ux65e5ux8a8c0041-20191108}}

\hypertarget{ux65e5ux671f-date-7}{%
\subsection{日期 Date}\label{ux65e5ux671f-date-7}}

\begin{itemize}
\tightlist
\item
  世界協調時間2019年(中華民國108年,令和1年)11月8日 / Unix 紀元 18208 日
  / 星期五 / 蒼白球紀元第41日
\item
  November 08, 2019 (UTC) / 18208 days since Unix Epoch / Friday /
  Globus Pallidum day 41
\item
  特殊註記:
\end{itemize}

\hypertarget{ux5e74ux9f61-age-7}{%
\subsection{年齡 Age}\label{ux5e74ux9f61-age-7}}

\begin{itemize}
\tightlist
\item
  33 years 6 months 16 days old / 2 years 0 months 27 days after
  acquiring ROC Surgical Pathology Licence
\item
  33 歲 6 個月 16 天 / 成為病理專科醫師 2 年 0 個月 27 天
\end{itemize}

\hypertarget{ux672cux6587-content-7}{%
\subsection{本文 Content}\label{ux672cux6587-content-7}}

\begin{enumerate}
\def\labelenumi{\arabic{enumi}.}
\item
  食記{[}1{]}

  中午從H大坐公車回家裡拿小提琴的過程中,需要從地方法院轉車。由於下一班車要12分鐘才到,因此在附近找看看有沒有什麼東西可以填肚子當午餐吃。放眼望去,發現這個地方居然沒有便利商店,正在絕望的時候,突然聞到不起眼的小店冒出好香好香的麵包味道,忍不住就走進去了。

  結果居然是大名店「洪瑞珍餅店」的其中一家分店。也開得太低調了,莫非老台中的貴族店都這麼樸實無華嗎?

  全麥火腿三明治30元,坦白講實在不便宜,但名店就是名店,咬下去的口感就是比平常的火腿三明治豐富許多。全中華民國國民都願意從台中訂購這家的三明治,真的是有原因的。配一瓶瑞穗鮮奶32元。

  晚餐:「丼脈」(因為老闆似乎有醫療背景,所以念法應該是「動脈」,取artery之意)薑汁牛丼,「李家紅茶冰」麥香紅茶,不寫價格因為C院請客。
\item
  與學弟妹還有母后一起團練後

  因為有一位學妹需要拿大提琴,所有人都需要拿譜架,所以母后就自告奮勇,11月11日的演出要開車載學弟妹去演出場地。我一方面覺得感謝他願意幫忙這場演出,另一方面又覺得,三十幾歲了這種事情還要老母幫忙,真的在學弟妹面前很沒面子。

  或許只能拿男友曾經講過的話安慰自己:「反正很多人都是靠著別人的善意才得以生存的」。
\item
  賀!

  睡超飽!覺得智商變成了1.5倍!
\end{enumerate}

\hypertarget{ux6ce8ux91cb-comment-7}{%
\subsection{注釋 Comment}\label{ux6ce8ux91cb-comment-7}}

{[}1{]} 新台幣計價。有關新台幣請參見蒼白球日誌0007。

{[}2{]}
引述店家簡介:洪瑞珍餅店自由店成立於民國70年,負責人洪幸雄13歲就投入烘培業,一路從彰化北斗洪瑞珍創始店,
二林店,
台中中山店一直到現在已逾70歲仍然堅守在洪瑞珍烘培師的領域,一步一腳印腳踏實地的做好烘培者的角色。「自由店」的產品都很樸實不走花俏但很堅持產品的內涵,有如負責人的個性一樣,堅持用有品牌有信譽的原料也是自由店的堅持,產品所需的內餡也都依最傳統費時的方式來蒸煮自製這是保護消費者最基本的態度。

\hypertarget{ux9644ux9304-appendix-7}{%
\subsection{附錄 Appendix}\label{ux9644ux9304-appendix-7}}

\hypertarget{ux84bcux767dux7403ux65e5ux8a8c0042-20191109}{%
\section{蒼白球日誌0042
2019/11/09}\label{ux84bcux767dux7403ux65e5ux8a8c0042-20191109}}

\hypertarget{ux65e5ux671f-date-8}{%
\subsection{日期 Date}\label{ux65e5ux671f-date-8}}

\begin{itemize}
\tightlist
\item
  世界協調時間2019年(中華民國108年,令和1年)11月9日 / Unix 紀元 18209 日
  / 星期六 / 蒼白球紀元第42日
\item
  November 09, 2019 (UTC) / 18209 days since Unix Epoch / Saturday /
  Globus Pallidum day 42
\end{itemize}

\hypertarget{ux5e74ux9f61-age-8}{%
\subsection{年齡 Age}\label{ux5e74ux9f61-age-8}}

\begin{itemize}
\tightlist
\item
  33 years 6 months 17 days old / 2 years 0 months 28 days after
  acquiring ROC Surgical Pathology Licence
\item
  33 歲 6 個月 17 天 / 成為病理專科醫師 2 年 0 個月 28 天
\end{itemize}

\hypertarget{ux672cux6587-content-8}{%
\subsection{本文 Content}\label{ux672cux6587-content-8}}

\begin{enumerate}
\def\labelenumi{\arabic{enumi}.}
\item
  數學與對男友情緒勒索

  因為某個契機,突然發現自己並沒有高中畢業應有的基本數學素養。在質疑自己的中華民國公民資格的同時,玻璃心{[}1{]}就碎掉了,於是這兩天強迫男友聽了一堆有的沒有的屁話。

  在情緒稍微平復一點之後,回頭再看一次那些話,驚覺我把自信心問題跟有的沒有的壓力都發洩在別人身上,完全就是一種很糟糕的情緒勒索。不過,做這個動作真是很舒壓,做完心情就平復了一半。另外一半則是開始寫這段文字的時候開始平復的,因為知道寫下來會有人看,舒壓的速度又更快了。我畢竟是一個非常怕孤單的人,所以用這種不太合理的方式尋求陪伴跟傾聽,請大家見諒。

  再次感謝男友跟其他讀者。然後我該來讀微積分跟統計了(棒棒的男友說很樂意教我),生活在這世界上總是要有一點數感比較好。
\item
  跟母后一起去教非洲鼓{[}2{]}

  今天的課程很奇妙,是某個連鎖餐飲店的員工訓練,原因是「老闆希望員工有一些人文素養」。其實人文素養這個東西是什麼,本身就是一個很困難的問題,但是因為這個名詞太炫麗了,於是就成為了中華民國國民做各種自認很有氣質的假掰{[}3{]}事的藉口。至於人文素養裡面有什麼,好像就沒有人在乎了。

  結果這個,嗯,因為老闆認為學音樂增進氣質所以開辦的非洲鼓訓練,最後招到了一些原本就有學音樂的員工來參加,好像做了什麼但是又等於什麼都沒做。現實世界總是比小說還荒謬。

  嗯但是每天寫廢話,這個月底還想要把這些廢話印成書{[}4{]}的我,好像也沒什麼理由批評別人荒謬。
\item
  雜記:物價與其他{[}5{]}

  \begin{itemize}
  \tightlist
  \item
    「麥味登」青醬燻雞義大利麵80元(大地雷!千萬不要去麥味登吃義大利麵!),「李記紅茶冰」25元
  \item
    晚餐回濁水溪老家陪阿嬤跟姑姑們吃祈福的桌菜,所以沒花自己的錢,但是過程很折磨人,因為所有長輩都對現況還有未來大局有非常悲觀的看法,且直接認為選了民進黨政府上台的錯。這會很折磨人的原因,一方面是由於害怕共產黨,所以不管民進黨政府做了多少髒事,我還是偏向支持它,立場差異是第一個不開心的地方。另一方面,我自承對政府運作的了解有限,不可能理解這些複雜的議題,所以沒辦法像某些綠營的狂信者那樣,用膝反射回應膝反射,只能沉默,無話可說是第二個不開心的地方。而在沉默與沉默與壓抑與壓抑之間,唯一的救贖或許只剩下蒼白球日誌。寫下來的事情才能夠釋放,寫下來的事情才能夠遺忘,寫下來的事情才能夠回想。
  \end{itemize}
\end{enumerate}

\hypertarget{ux6ce8ux91cb-comment-8}{%
\subsection{注釋 Comment}\label{ux6ce8ux91cb-comment-8}}

{[}1{]}
玻璃易碎,玻璃心是形容人自尊心極容易受損、心靈異常脆弱、極容易心碎的意思。

{[}2{]} 非洲鼓與我們的非洲鼓課程,請見蒼白球日誌0005。

{[}3{]} 原自閩南語「膣屄」,指一個人思想或行為很做作。

{[}4{]}
暫名「蒼白球日誌2019秋(0001-0063)」,原本想要找C市的小量印刷店印,離工作地點近比較方便,但是看到P市的樺舍有提供一本試印的服務就心動了,到時應該會找樺舍印吧。

{[}5{]} 新台幣計價。有關新台幣請參見蒼白球日誌0007。

\hypertarget{ux9644ux9304-appendix-8}{%
\subsection{附錄 Appendix}\label{ux9644ux9304-appendix-8}}

\hypertarget{ux84bcux767dux7403ux65e5ux8a8c0043-20191110}{%
\section{蒼白球日誌0043
2019/11/10}\label{ux84bcux767dux7403ux65e5ux8a8c0043-20191110}}

\hypertarget{ux65e5ux671f-date-9}{%
\subsection{日期 Date}\label{ux65e5ux671f-date-9}}

\begin{itemize}
\tightlist
\item
  世界協調時間2019年(中華民國108年,令和1年)11月10日 / Unix 紀元 18210
  日 / 星期日 / 蒼白球紀元第43日
\item
  November 10, 2019 (UTC) / 18210 days since Unix Epoch / Sunday /
  Globus Pallidum day 43
\end{itemize}

\hypertarget{ux5e74ux9f61-age-9}{%
\subsection{年齡 Age}\label{ux5e74ux9f61-age-9}}

\begin{itemize}
\tightlist
\item
  33 years 6 months 18 days old / 2 years 0 months 29 days after
  acquiring ROC Surgical Pathology Licence
\item
  33 歲 6 個月 18 天 / 成為病理專科醫師 2 年 0 個月 29 天
\end{itemize}

\hypertarget{ux672cux6587-content-9}{%
\subsection{本文 Content}\label{ux672cux6587-content-9}}

\begin{enumerate}
\def\labelenumi{\arabic{enumi}.}
\item
  音樂欣賞簡報

  母后下周六日要去教一堂七小時的課程,內容有音樂欣賞以及非洲鼓,我今天待在家裡幫他做教材。

  之所以非得我做不可,其中一個原因是母后的課程通常需要很多影片輔助,因此必須去Youtube上面搜尋適合的影片{[}1{]},並且盜拷下來。這個操作對於母后這個年齡的人來說相當吃力,所以都是我在幫忙。另外還有一點就是,有很多有的沒有的創意也只有我可以想得出來。

  簡而言之,母后近年的教學事業其實都是我在背後輔助。其實滿累的。
\item
  文學與政治

  一個時代的文字環境總是與政權息息相關,這無可厚非,文字本來就是一種很政治的工具,但是最近包含朱宥勳激怒駱以軍事件{[}2{]}在內的很多事情,讓我覺得繁體中文文字域的政權味有點太過份了。有點太難忍受了。當然,這個情形當然是本來就很有可能發生的沒錯,畢竟社群媒體能夠無限放大文字的力量,必然造成政黨被迫對這個文字域進行操作,造成人們的文字域裡面逐漸充滿了飽和的,單一的,政黨政治的元素。

  但是仔細想想又很怪。

  我們有人類歷史以來最豐富的文字發表環境,最低的文字發表成本,結果文字域裡面卻填滿了很無聊單一的東西,這實在很弔詭又很討厭。因此我主張人們應該用各式各樣不同的語言文字抵抗政黨的汙染。像韓少功所主張的一樣,每個人都要編一套屬於自己的辭典。
\end{enumerate}

\hypertarget{ux6ce8ux91cb-comment-9}{%
\subsection{注釋 Comment}\label{ux6ce8ux91cb-comment-9}}

{[}1{]}
YouTube是源自美國的影片分享網站,讓使用者上傳、觀看、分享及評論影片。公司於2005年2月14日註冊,網站的口號為「Broadcast
Yourself」,網站的標誌意念來自早期電視顯示器。
目前尚無官方的中文譯名,較為廣泛使用的俗稱有油管、水管、你管等。

{[}2{]}
朱宥勳(1988年1月4日-),臺灣桃園人,小說家、文化評論者、專欄作家,國立清華大學人文社會學系學士(主修社會學、歷史)、國立清華大學台灣文學研究所碩士。駱以軍(1967年3月29日-),台北縣(今新北市)人,籍貫安徽省無為縣,臺灣專職作家。2019年11月5日,朱宥勳在個人臉書公開發表駱以軍新長篇小說《明朝》的書評,因書評的政黨色彩而激怒駱以軍,從而掀起了一場社群媒體論戰。

\hypertarget{ux9644ux9304-appendix-9}{%
\subsection{附錄 Appendix}\label{ux9644ux9304-appendix-9}}

\hypertarget{ux84bcux767dux7403ux65e5ux8a8c0044-20191111}{%
\section{蒼白球日誌0044
2019/11/11}\label{ux84bcux767dux7403ux65e5ux8a8c0044-20191111}}

\hypertarget{ux65e5ux671f-date-10}{%
\subsection{日期 Date}\label{ux65e5ux671f-date-10}}

\begin{itemize}
\tightlist
\item
  世界協調時間2019年(中華民國108年,令和1年)11月11日 / Unix 紀元 18211
  日 / 星期一 / 蒼白球紀元第44日
\item
  November 11, 2019 (UTC) / 18211 days since Unix Epoch / Monday /
  Globus Pallidum day 44
\end{itemize}

\hypertarget{ux5e74ux9f61-age-10}{%
\subsection{年齡 Age}\label{ux5e74ux9f61-age-10}}

\begin{itemize}
\tightlist
\item
  33 years 6 months 19 days old / 2 years 0 months 30 days after
  acquiring ROC Surgical Pathology Licence
\item
  33 歲 6 個月 19 天 / 成為病理專科醫師 2 年 0 個月 30 天
\end{itemize}

\hypertarget{ux672cux6587-content-10}{%
\subsection{本文 Content}\label{ux672cux6587-content-10}}

\begin{enumerate}
\def\labelenumi{\arabic{enumi}.}
\item
  醫師節院內宴會,與醫學生還有母后進行了一場成功的鋼琴-弦樂重奏演出

  之後演出人員坐在同一桌吃這場宴會。如果你以為跟年輕的帥哥美女同桌吃飯是享受,那就大錯特錯了。這是一種最糟糕的應酬,惡夢的宴席。由於主治醫師有權力打醫學生的成績,且未來應徵住院醫師的時候必定會過主治醫師這一關,醫學生在這場飯局中用各種業務般油滑的話術對我各種裝熟,刺探醫院的各種秘辛,還要求合照,甚至是要Instagram{[}1{]}帳號,簡直像在談生意一樣。

  不舒服的除了那些很油很刻意的話術以外,還有那個權力不對等的氣氛。被當成需要加以奉承的長官,對我來講不僅非常怪,而且更嚴重的是,覺得變老了。幸好我下禮拜就擺脫這幾個人了,如果長期要跟醫學生一起合奏的話,我一定會被這些業務嘴煩到瘋掉。

  然後要Instagram的要求被我回絕了。我不想要跟醫學生有太多私交。
\item
  雜記:物價與其他{[}2{]}

  C院午餐72元(紫米飯,洋蔥豬柳,地瓜葉,青花菜,西洋芹),「泰山」氣泡水兩瓶39元(第二瓶十元),醫師節晚宴免費,用C院禮券買的一堆飲料104元(禮券100元,補4元用悠遊卡付)。
\end{enumerate}

\hypertarget{ux6ce8ux91cb-comment-10}{%
\subsection{注釋 Comment}\label{ux6ce8ux91cb-comment-10}}

{[}1{]}
Instagram是Facebook公司旗下一款免費提供線上圖片及視訊分享的社交應用軟體,於2010年10月發布。它可以讓用戶用智慧型手機拍下相片後再將不同的濾鏡效果添加到相片上,然後分享到Facebook、Twitter、Tumblr及Flickr等社群網路服務、或是Instagram的伺服器上。

{[}2{]} 新台幣計價。有關新台幣請參見蒼白球日誌0007。

\hypertarget{ux9644ux9304-appendix-10}{%
\subsection{附錄 Appendix}\label{ux9644ux9304-appendix-10}}

\hypertarget{ux84bcux767dux7403ux65e5ux8a8c0045-20191112}{%
\section{蒼白球日誌0045
2019/11/12}\label{ux84bcux767dux7403ux65e5ux8a8c0045-20191112}}

\hypertarget{ux65e5ux671f-date-11}{%
\subsection{日期 Date}\label{ux65e5ux671f-date-11}}

\begin{itemize}
\tightlist
\item
  世界協調時間2019年(中華民國108年,令和1年)11月12日 / Unix 紀元 18212
  日 / 星期二 / 蒼白球紀元第45日
\item
  November 12, 2019 (UTC) / 18212 days since Unix Epoch / Tuesday /
  Globus Pallidum day 45
\item
  特殊註記:
\end{itemize}

\hypertarget{ux5e74ux9f61-age-11}{%
\subsection{年齡 Age}\label{ux5e74ux9f61-age-11}}

\begin{itemize}
\tightlist
\item
  33 years 6 months 20 days old / 2 years 1 months 0 days after
  acquiring ROC Surgical Pathology Licence
\item
  33 歲 6 個月 20 天 / 成為病理專科醫師 2 年 1 個月 0 天
\end{itemize}

\hypertarget{ux672cux6587-content-11}{%
\subsection{本文 Content}\label{ux672cux6587-content-11}}

\begin{enumerate}
\def\labelenumi{\arabic{enumi}.}
\item
  一罐濃縮時間

  雖然我也知道年過三十要保持青春是不可能的,但是升主治醫師這兩年來,我的外貌跟心智實在是老化得太過頭了。診斷的壓力、研究的壓力、科內的種種變化,讓人好像喝下了一罐濃縮時間一樣,不斷往衰敗的方向前進。

  更可怕的是,這不是那種在歲月的推進中不知不覺變老,而是每一天每一天都覺得被時間追著跑的變老,不只喝下濃縮時間,喝下的過程還每一刻都在舌尖感覺到濃縮時間的苦味。而這就是我對這個日誌如此執著的原因。被灌食這麼苦的濃縮時間,總要結晶出什麼時間的舍利子,不然豈不是太虧了嗎?
\item
  母后送我的高檔防曬乳沒了,於是我跑去添購

  可能是因為我身上散發出的魯味{[}1{]}太濃吧,一到Elisabeth
  Arden的專櫃前面,專櫃店員就很明顯地出現冷淡的表情,完全不想理我。直到我提出要求說要買那管黃色的防曬乳,才吐出一句淡淡的「是幫別人買的嗎?」她心中想到的故事想必是以下這款:

  A君,33歲,幾年前花了家裡的積蓄勉強念到研究所畢業以後,因為體格不好做不了勞力工作,只好為了養活老婆小孩到處打不用體力的工。有一天老婆突然吵說他想要高檔保養品,只好硬擠出了一點錢,走到了從來不敢接近的專櫃前面\ldots\ldots{}

  他絕對想不到我是情形截然不同的33歲B君。算了。

  當我說出是我自己要用的時候,店員才露出了有點詫異的表情,然後有點錯愕地幫我結帳了。我以後可能需要時時霸氣地宣告「林北是主治醫師」才能夠避免被當成魯蛇吧\ldots\ldots 但那又很怪。
\item
  雜記:物價與其他{[}2{]}

  \begin{itemize}
  \tightlist
  \item
    Arden防曬乳,1300元。
  \item
    Tunemaker的一堆保養品,2072元,購於康是美(原價好像2350左右,結果店員因為超過兩千可以免費辦會員卡,就幫我辦了一張,在一路發動許多優惠之後不只折價到2072,還多帶了一罐保養品回家(因為EGF買一送一))
  \item
    C院午餐,72元
    (紫米飯,排骨,白花菜,西洋芹,杏鮑菇),一中街某無名店面的腸粉100元(店長看起來是普通的中年婦女,陪小孩在小小舊舊的店裡面做功課,然後小孩還耍脾氣,媽媽一邊安撫小孩一邊蒸我的腸粉,看起來很魯。但這說不定只是錯覺,這店在一中街精華地帶欸,他搞不好賺很多錢。有很多看起來魯的東西不一定真魯,就像我一樣。),「出櫃
    」蕎麥冰茶20元。
  \end{itemize}
\end{enumerate}

\hypertarget{ux6ce8ux91cb-comment-11}{%
\subsection{注釋 Comment}\label{ux6ce8ux91cb-comment-11}}

{[}1{]}
魯蛇(英語:loser),又稱魯者男、擼蛇、廢青、輸家男等,是大部分東亞地區網絡的一種諷刺語,意即「人生的失敗者」,最早在1993年由韓國匿名網民創設,在網絡作為隱語流通,2012年左右開始有華人網友使用此用法,於是逐漸在華人地區流行。魯味,即魯蛇的味道。

{[}2{]} 新台幣計價。有關新台幣請參見蒼白球日誌0007。

{[}3{]}
康是美藥妝店,簡稱康是美,是台灣一家連鎖藥妝店。截至2018年8月,已開設402家直營門市,在中國大陸亦有店鋪經營。
在台灣的主要競爭對手為屈臣氏。
2018年康是美營業額110.27億新臺幣,盈利2.9億新臺幣,總資產43.98億,負債30.31億,淨資產13.67億新臺幣。

\hypertarget{ux9644ux9304-appendix-11}{%
\subsection{附錄 Appendix}\label{ux9644ux9304-appendix-11}}

\hypertarget{ux84bcux767dux7403ux65e5ux8a8c0046-20191113}{%
\section{蒼白球日誌0046
2019/11/13}\label{ux84bcux767dux7403ux65e5ux8a8c0046-20191113}}

\hypertarget{ux65e5ux671f-date-12}{%
\subsection{日期 Date}\label{ux65e5ux671f-date-12}}

\begin{itemize}
\tightlist
\item
  世界協調時間2019年(中華民國108年,令和1年)11月13日 / Unix 紀元 18213
  日 / 星期三 / 蒼白球紀元第46日
\item
  November 13, 2019 (UTC) / 18213 days since Unix Epoch / Wednesday /
  Globus Pallidum day 46
\item
  特殊註記:
\end{itemize}

\hypertarget{ux5e74ux9f61-age-12}{%
\subsection{年齡 Age}\label{ux5e74ux9f61-age-12}}

\begin{itemize}
\tightlist
\item
  33 years 6 months 21 days old / 2 years 1 months 1 days after
  acquiring ROC Surgical Pathology Licence
\item
  33 歲 6 個月 21 天 / 成為病理專科醫師 2 年 1 個月 1 天
\end{itemize}

\hypertarget{ux672cux6587-content-12}{%
\subsection{本文 Content}\label{ux672cux6587-content-12}}

\begin{enumerate}
\def\labelenumi{\arabic{enumi}.}
\item
  「明朝」{[}1{]}

  在買下紹中的「在流放地」同時,我也下訂了「明朝」這本爭議之作。翻開前幾節的時候其實非常困惑,不是說這是小說嗎?怎麼會結構如此鬆散,幾乎沒有在正經說故事呢?而且,不時就跳到作者對古今文化的評論,非常出戲。

  直到我開始利用零碎時間把這本書跳著看以後,驚覺,啊,這本本來就不是小說。他其實是藉由一個科幻小說設定的軀殼,寫各種文化評論跟台灣社會的隨筆。所以我試著直接把那些看起來很不自然的情節拋在腦後,讀剩下的散文部分在腦中浮現,意外地覺得讀起來還算有意思。難怪某警犬會說「我覺得這本某個程度很適合你的調性」「讀不完但有可讀性」,某醫學生會說「我覺得你可以讀讀看」。

  這本坦白講絕對不會成為名著,但應該是有它的價值存在的。
\item
  與男友炫耀論文被引用

  講的時候放不下矜持,一直說什麼哎呀其實很意外人家寫大作會引用到這麼爛的期刊,什麼其實這個也沒啥創見只是剛好印證了某大型機構的資料,等等謙抑的說詞。炫耀完以後才驚覺,其實心裡根本就是爽翻天,幹嘛還要憋著,講一堆那麼收斂的說詞呢?

  應該要放聲大笑才對。哇哈哈哈哈哈,哇哈哈哈哈哈,哇哈哈哈哈哈,我的論文被Cell{[}2{]}上面的文章引用了!超爽的!超爽的!超爽的!
\item
  雜記:物價與其他{[}3{]}

  \begin{itemize}
  \tightlist
  \item
    C院午餐72元(紫米飯,排骨、青白混合花菜、菠菜、然後忘記一種菜),「全家便利商店」腿排便當44元(原價62元,友善食光7折{[}4{]}),「雀巢」黑糖奶茶30元(上健身教練克前補一點糖),十顆抗組織胺75元,「清心福全」普洱25元
  \end{itemize}
\end{enumerate}

\hypertarget{ux6ce8ux91cb-comment-12}{%
\subsection{注釋 Comment}\label{ux6ce8ux91cb-comment-12}}

{[}1{]} 指駱以軍的書。有關此書與其掀起的爭議,請見蒼白球日誌0043。

{[}2{]}
《細胞》為一份同行評審科學期刊,主要發表生命科學領域中的最新研究發現。《細胞》刊登過許多重大的生命科學研究進展,與《自然》和《科學》並列,是全世界最權威的學術雜誌之一。

{[}3{]} 新台幣計價。有關新台幣請參見蒼白球日誌0007。

{[}4{]} 快要過保存期限的熟食打折販賣

\hypertarget{ux9644ux9304-appendix-12}{%
\subsection{附錄 Appendix}\label{ux9644ux9304-appendix-12}}

\begin{verbatim}
圖:2019/11/13台中中山堂旁的滿月 https://imgur.com/gallery/WPl9ww3
註:難得台中的空氣乾淨到能夠看到這麼亮的滿月,不禁覺得充滿希望,讓人想起高橋禮子在經典歌曲METHOD_REPLEKIA/.中寫下的詩句:
xA rre wArAmA maen a.u.k. zess titia/. (黑暗中皎潔光亮的滿月升起,顯現了聖潔與公義)
xE rre hAkAtt nafan ouwua siann arhou/. (月光放出千萬個希望,照耀人間)
xA rre sEnEkk mirie, ag hEmmrA eje/. (而心之歌就如月光般溫暖光亮)
\end{verbatim}

\hypertarget{ux84bcux767dux7403ux65e5ux8a8c0047-20191114}{%
\section{蒼白球日誌0047
2019/11/14}\label{ux84bcux767dux7403ux65e5ux8a8c0047-20191114}}

\hypertarget{ux65e5ux671f-date-13}{%
\subsection{日期 Date}\label{ux65e5ux671f-date-13}}

\begin{itemize}
\tightlist
\item
  世界協調時間2019年(中華民國108年,令和1年)11月14日 / Unix 紀元 18214
  日 / 星期四 / 蒼白球紀元第47日
\item
  November 14, 2019 (UTC) / 18214 days since Unix Epoch / Thursday /
  Globus Pallidum day 47
\end{itemize}

\hypertarget{ux5e74ux9f61-age-13}{%
\subsection{年齡 Age}\label{ux5e74ux9f61-age-13}}

\begin{itemize}
\tightlist
\item
  33 years 6 months 22 days old / 2 years 1 months 2 days after
  acquiring ROC Surgical Pathology Licence
\item
  33 歲 6 個月 22 天 / 成為病理專科醫師 2 年 1 個月 2 天
\end{itemize}

\hypertarget{ux672cux6587-content-13}{%
\subsection{本文 Content}\label{ux672cux6587-content-13}}

\begin{enumerate}
\def\labelenumi{\arabic{enumi}.}
\item
  路上遇到曾經跟我們科關係密切的院內資深學者J博士

  她驚訝於我在一兩個月內變得非常憔悴的面容,不禁問:「發生什麼事了?」

  於是我開始訴說起再研究計畫核銷跟臨床工作的壓力下,又加上兩場院慶演出對我造成的重大負擔。在聽到這些以後,J博士馬上傳授我研究計畫核銷的各種秘訣,最後丟下兩句話:「自己身體要顧好。要懂得閃一些事情。」

  我會謹記在心的。
\item
  一直在想蒼白球日誌要直印還是橫印的問題

  於是翻閱了我手上有的書籍,基本上散文或小說絕大多數都是直印,閱讀起來比較舒服,唯有一本例外:蒼子的《潮間帶》{[}1{]}。

  所以我稍微又把這本翻了一下,其實也沒有因為橫印而難閱讀,那麼我還是橫印好了,對LaTeX{[}2{]}來說比較方便。
\item
  雜記:物價與其他{[}3{]}

  C院便當72元(因為餐廳要慶生會所以只賣預配好的便當不賣自助餐,菜色很糟糕,有我最討厭的糖醋豆皮,任何東西加了糖醋或美乃滋都會瞬間變得難以下嚥,不知道發明這兩種東西的人類在想什麼),「泰山」氣泡水兩瓶39元,「韓石館」石頭牛雜鍋145元(好吃且好飽)
\end{enumerate}

\hypertarget{ux6ce8ux91cb-comment-13}{%
\subsection{注釋 Comment}\label{ux6ce8ux91cb-comment-13}}

{[}1{]} 我與此書的因緣請見蒼白球日誌0037。

{[}2{]} 有關LaTeX請見蒼白球日誌0028。

{[}3{]} 新台幣計價。有關新台幣請參見蒼白球日誌0007。

\hypertarget{ux9644ux9304-appendix-13}{%
\subsection{附錄 Appendix}\label{ux9644ux9304-appendix-13}}

\hypertarget{ux84bcux767dux7403ux65e5ux8a8c0048-20191115}{%
\section{蒼白球日誌0048
2019/11/15}\label{ux84bcux767dux7403ux65e5ux8a8c0048-20191115}}

\hypertarget{ux65e5ux671f-date-14}{%
\subsection{日期 Date}\label{ux65e5ux671f-date-14}}

\begin{itemize}
\tightlist
\item
  世界協調時間2019年(中華民國108年,令和1年)11月15日 / Unix 紀元 18215
  日 / 星期五 / 蒼白球紀元第48日
\item
  November 15, 2019 (UTC) / 18215 days since Unix Epoch / Friday /
  Globus Pallidum day 48
\end{itemize}

\hypertarget{ux5e74ux9f61-age-14}{%
\subsection{年齡 Age}\label{ux5e74ux9f61-age-14}}

\begin{itemize}
\tightlist
\item
  33 years 6 months 23 days old / 2 years 1 months 3 days after
  acquiring ROC Surgical Pathology Licence
\item
  33 歲 6 個月 23 天 / 成為病理專科醫師 2 年 1 個月 3 天
\end{itemize}

\hypertarget{ux672cux6587-content-14}{%
\subsection{本文 Content}\label{ux672cux6587-content-14}}

\begin{enumerate}
\def\labelenumi{\arabic{enumi}.}
\item
  聽蛋白質結構的課

  老師提到最近四十年已經沒有人在手算蛋白質立體結構了,也不再製作實體模型,都是用電腦建模,然後儲存在線上資料庫。這讓我突然發現一件事情,當人類失去電力的時候,蛋白質的知識會馬上化為泡影,實在是很悲傷。

  然而人類其實沒有什麼方式可以處理這個問題,因為蛋白質立體結構複雜到紙本難以描述。或許只能過一天算一天,在電力允許的範圍內盡量維護現有的蛋白質資料庫。
\item
  我的粉專{[}2{]}在醫學生之間有點太過有名

  而造成了一些困擾。在病理部這梯醫學生結訓座談上面,當我跟醫學生因為無話可談而陷入沉默的時候,他們居然用「老師要多貼粉專喔!我們都有在看!」做結語結束這個座談,被這樣套交情實在太尷尬了。這種會模糊師生界線的工具實在是必須慎用。
\item
  雜記:物價與其他{[}3{]}

  幫母后線上買譜,兩份共170元,「麥當勞」大麥克餐122元,印譜(約70頁,訂騎馬釘)68元。
\end{enumerate}

\hypertarget{ux6ce8ux91cb-comment-14}{%
\subsection{注釋 Comment}\label{ux6ce8ux91cb-comment-14}}

{[}1{]} Facebook{[}2{]}
企業粉絲專頁是公開觸及桌上型電腦和手機用戶粉絲群的免費管道。Facebook
企業粉絲專頁是為企業、品牌、名人、公益理念和團體組織打造的專屬空間。Google
可能會將您的粉絲專頁編入索引,讓其他人更更容易找到您的企業。請使用桌上型電腦、手機以及專頁小助手應用程式來管理粉絲專頁。

{[}2{]}
Facebook(簡稱FB)是源於美國的社群網路服務及社會化媒體網站,總部位於美國加州聖馬刁郡門洛公園市。成立初期原名為「thefacebook」,名稱的靈感來自美國高中提供給學生包含相片和聯絡資料的通訊錄(或稱花名冊)之暱稱「face
book」。目前尚無官方的中文譯名,較為廣泛使用則為臉書。

{[}3{]} 新台幣計價。有關新台幣請參見蒼白球日誌0007。

\hypertarget{ux9644ux9304-appendix-14}{%
\subsection{附錄 Appendix}\label{ux9644ux9304-appendix-14}}

\hypertarget{ux84bcux767dux7403ux65e5ux8a8c0049-20191116}{%
\section{蒼白球日誌0049
2019/11/16}\label{ux84bcux767dux7403ux65e5ux8a8c0049-20191116}}

\hypertarget{ux65e5ux671f-date-15}{%
\subsection{日期 Date}\label{ux65e5ux671f-date-15}}

\begin{itemize}
\tightlist
\item
  世界協調時間2019年(中華民國108年,令和1年)11月16日 / Unix 紀元 18216
  日 / 星期六 / 蒼白球紀元第49日
\item
  November 16, 2019 (UTC) / 18216 days since Unix Epoch / Saturday /
  Globus Pallidum day 49
\item
  特殊註記:
\end{itemize}

\hypertarget{ux5e74ux9f61-age-15}{%
\subsection{年齡 Age}\label{ux5e74ux9f61-age-15}}

\begin{itemize}
\tightlist
\item
  33 years 6 months 24 days old / 2 years 1 months 4 days after
  acquiring ROC Surgical Pathology Licence
\item
  33 歲 6 個月 24 天 / 成為病理專科醫師 2 年 1 個月 4 天
\end{itemize}

\hypertarget{ux672cux6587-content-15}{%
\subsection{本文 Content}\label{ux672cux6587-content-15}}

早上去院慶演出,接著去跟母后一起教非洲鼓,然後回家幫母后準備明天的教案。因為這樣真的累所以今天不想寫什麼。

\hypertarget{ux6ce8ux91cb-comment-15}{%
\subsection{注釋 Comment}\label{ux6ce8ux91cb-comment-15}}

\hypertarget{ux9644ux9304-appendix-15}{%
\subsection{附錄 Appendix}\label{ux9644ux9304-appendix-15}}

\hypertarget{ux84bcux767dux7403ux65e5ux8a8c0050-20191117}{%
\section{蒼白球日誌0050
2019/11/17}\label{ux84bcux767dux7403ux65e5ux8a8c0050-20191117}}

\hypertarget{ux65e5ux671f-date-16}{%
\subsection{日期 Date}\label{ux65e5ux671f-date-16}}

\begin{itemize}
\tightlist
\item
  世界協調時間2019年(中華民國108年,令和1年)11月17日 / Unix 紀元 18217
  日 / 星期日 / 蒼白球紀元第50日
\item
  November 17, 2019 (UTC) / 18217 days since Unix Epoch / Sunday /
  Globus Pallidum day 50
\item
  特殊註記:
\end{itemize}

\hypertarget{ux5e74ux9f61-age-16}{%
\subsection{年齡 Age}\label{ux5e74ux9f61-age-16}}

\begin{itemize}
\tightlist
\item
  33 years 6 months 25 days old / 2 years 1 months 5 days after
  acquiring ROC Surgical Pathology Licence
\item
  33 歲 6 個月 25 天 / 成為病理專科醫師 2 年 1 個月 5 天
\end{itemize}

\hypertarget{ux672cux6587-content-16}{%
\subsection{本文 Content}\label{ux672cux6587-content-16}}

\begin{enumerate}
\def\labelenumi{\arabic{enumi}.}
\item
  人跟人之間的共振(本來對自己說好不寫的,但是不寫實在太悶了,只好寫出來)

  昨天又去了一場全桌一齊數落民進黨政府的應酬,真的非常悶。悶的原因是,一方面我在明年大選傾向支持民進黨,另一方面是嫉妒這群人能夠彼此達成如此高度的共鳴,突然覺得寂寞。跟某神人講了這件事之後,他提醒我,相同製程製造的人類,就會有相近的頻率,因此發生共振是非常自然的。

  於是我突然想起了和聲學。

  大三和弦之所以會聽起來廣大而和諧,原因是這三個音的泛音列達成高度重疊,而出現大幅度的共振。被調成同一個大三和絃的人們,遇到政治事件的時候就會聚在一起共振,產生廣大且單調的聲音。目前台灣的民粹政治就是調音的藝術,兩大陣營互相比賽如何將自己的支持者調成自己的調性,消滅異端調性,直到聽起來像是一個單調的和絃為止。因此,我們會看到各種無益於實際政治運作的,莫名其妙的共振。

  而在這些互相競賽的交響樂團裡面,都找不到共振的我,到底是什麼呢?我屬於哪一個調性?莫非我在十幾年來的各種個人矛盾裡面,已經走音,跑到十二平均律{[}1{]}以外,無法歸類為任何調性了嗎?真的好落寞喔,這走音的感覺。
\item
  食記 {[}2{]}

  「萬代福排骨飯」{[}3{]}
  店面裝潢看起來非常舊,帶著1988年左右的泡沫經濟時代風格,且散發著濃濃的酸筍臭味,如果不是因為心情差想嘗鮮,真的完全不會走進這家店。

  本來以為自己會後悔,結果料理意外地相當不錯。排骨飯70元,竟然是裹著厚厚麵包粉跟一層味精炸的,拿到的時候覺得痘痘隱隱作痛了一下。對,一層味精,我甚至可以在表面看到那個針狀的結晶。在這種崇尚天然調味的世界,我得講一句政治不正確的話:那層味精的正是排骨美味的秘密!當滿滿肉汁的雪白豬肉、回鍋油、被炸成金色的麵包粉、還有滿滿味精,所有會被醫師譴責的邪惡風味物質一起帶著濃濃的健康詛咒入口的時候,只有「爽」一個字可以形容。一切最古老的、不健康的食慾好像都在這一口裡面爆發,對營養師齊聲喊了一句「幹你娘」。

  然後旁邊的白菜滷跟酸筍,怎麼說呢,簡直就是把排骨的爽字增幅到了好幾倍。白菜滷充滿了會讓營養師加倍尖叫的濃濃醬油味,然後酸筍就跟臭豆腐一樣,越臭就越香醇。店面聞起來很噁心,正是因為他們的酸筍真香。豬血湯20元。我愛豬血湯成癡,所以對豬血湯的評論可能不公允,姑且略過。

  中華路「40年老店青草茶」

  大杯50元,一句話爆苦。我不會再喝這家了。
\item
  雜記

  \begin{itemize}
  \tightlist
  \item
    滑輪下拉10下三組,機械胸推10下三組,二頭肌10下三組,三頭肌10下三組,跑步8分鐘。
  \item
    跟母后去教鼓很開心。只有在做音樂的時候我們之間的矛盾才會稍微緩解一點。
  \item
    母后看到我的臉很糟糕,送我幾條昂貴品牌的試用品,晚上用了以後隔天就馬上感覺到改善。原來對於沒有天生麗質的人而言,外貌也是需要花大錢才能擁有的東西嗎?世界真是不公平啊。
  \end{itemize}
\end{enumerate}

\hypertarget{ux6ce8ux91cb-comment-16}{%
\subsection{注釋 Comment}\label{ux6ce8ux91cb-comment-16}}

{[}1{]}
十二平均律,又稱十二等程律,是一種音樂的定律方法,將一個八度平均分成十二等份,每等分稱為半音,是最主要的調音法。音高八度音指的是頻率加倍(即二倍頻率)。八度音的頻率分為十二等分,即是分為十二項的等比數列,也就是每個音的頻率為前一個音的2的12次方根。19世紀以後的十二平均律,一般而言主音定為A1(440Hz)。

{[}2{]} 新台幣計價。有關新台幣請參見蒼白球日誌0007。

{[}3{]} 在C市中心著名的萬代福電影院旁邊,於是很乾脆的取成這個名字。

\hypertarget{ux9644ux9304-appendix-16}{%
\subsection{附錄 Appendix}\label{ux9644ux9304-appendix-16}}

\hypertarget{ux84bcux767dux7403ux65e5ux8a8c0051-20191118}{%
\section{蒼白球日誌0051
2019/11/18}\label{ux84bcux767dux7403ux65e5ux8a8c0051-20191118}}

\hypertarget{ux65e5ux671f-date-17}{%
\subsection{日期 Date}\label{ux65e5ux671f-date-17}}

\begin{itemize}
\tightlist
\item
  世界協調時間2019年(中華民國108年,令和1年)11月18日 / Unix 紀元 18218
  日 / 星期一 / 蒼白球紀元第51日
\item
  November 18, 2019 (UTC) / 18218 days since Unix Epoch / Monday /
  Globus Pallidum day 51
\item
  特殊註記:
\end{itemize}

\hypertarget{ux5e74ux9f61-age-17}{%
\subsection{年齡 Age}\label{ux5e74ux9f61-age-17}}

\begin{itemize}
\tightlist
\item
  33 years 6 months 26 days old / 2 years 1 months 6 days after
  acquiring ROC Surgical Pathology Licence
\item
  33 歲 6 個月 26 天 / 成為病理專科醫師 2 年 1 個月 6 天
\end{itemize}

\hypertarget{ux672cux6587-content-17}{%
\subsection{本文 Content}\label{ux672cux6587-content-17}}

\begin{enumerate}
\def\labelenumi{\arabic{enumi}.}
\item
  11月18日凌晨,香港警方圍困香港理工大學。

  原本只是將重要歷史事件記錄一下,以作為時間的定錨,不過因為突然想到村上春樹,所以順便補述一段,顯示自己很有學問。

  在小說「1973年的彈珠玩具」(村上春樹,1992)裡,有許多對於日本大型示威運動「全共鬥」的描寫。在這些片段中,最精彩的是其中警方攻陷東京大學的場景,回頭看起來,這段與香港理工大學的圍城衝突意外地相似。香港與東京,1969與2019,完全不同性質的政權,完全不同性質的人民,完全不同性質的時代,相互衝突的場景卻驚人地類似。歷史或許真的會重複吧。

  僅摘錄此書片段如下:
  「他屬於一個政治性的社團,那個社團佔據了大學的九號館。」
  「晴朗得令人愉快的十一月某個下午,當第三機動隊衝進九號館時,據說韋瓦第的``調和的靈感''正以全音量播出{[}1{]}。」
\item
  雜記:物價與其他{[}2{]}

  \begin{itemize}
  \tightlist
  \item
    五月我有一份報告打得有瑕疵,結果十一月病人復發,開刀檢體又落到我手上,回頭看那份報告覺得事情做壞了,心情不好。
  \item
    C院午餐72元(紫米飯,洋蔥豬柳,小白菜,杏鮑菇,菜豆,黃豆芽),「台灣第一味」甘蔗青茶50元,用C院院慶的兩百塊禮券去H棟吃了貴又難吃又鹹的麵店,因為太糟所以我不想紀錄吃了什麼。
  \end{itemize}
\end{enumerate}

\hypertarget{ux6ce8ux91cb-comment-17}{%
\subsection{注釋 Comment}\label{ux6ce8ux91cb-comment-17}}

{[}1{]} 調和的靈感(L'estro
armonico,定年1711年),義大利作曲家韋瓦第(Antonio
Vivaldi,1678-1741)為絃樂器所做的十二首協奏曲。村上春樹並未提及此處播放的是十二首中的哪一首、哪一段,但咸信是充滿氣魄,戰鬥般的第三號協奏曲第三樂章。對了,村上春樹介紹的音樂沒有不好聽的,基本上村上春樹的小說不只是小說,還是史書跟可靠的樂評。

{[}2{]} 新台幣計價。有關新台幣請參見蒼白球日誌0007。

\hypertarget{ux9644ux9304-appendix-17}{%
\subsection{附錄 Appendix}\label{ux9644ux9304-appendix-17}}

\hypertarget{ux84bcux767dux7403ux65e5ux8a8c0052-20191119}{%
\section{蒼白球日誌0052
2019/11/19}\label{ux84bcux767dux7403ux65e5ux8a8c0052-20191119}}

\hypertarget{ux65e5ux671f-date-18}{%
\subsection{日期 Date}\label{ux65e5ux671f-date-18}}

\begin{itemize}
\tightlist
\item
  世界協調時間2019年(中華民國108年,令和1年)11月19日 / Unix 紀元 18219
  日 / 星期二 / 蒼白球紀元第52日
\item
  November 19, 2019 (UTC) / 18219 days since Unix Epoch / Tuesday /
  Globus Pallidum day 52
\item
  特殊註記:
\end{itemize}

\hypertarget{ux5e74ux9f61-age-18}{%
\subsection{年齡 Age}\label{ux5e74ux9f61-age-18}}

\begin{itemize}
\tightlist
\item
  33 years 6 months 27 days old / 2 years 1 months 7 days after
  acquiring ROC Surgical Pathology Licence
\item
  33 歲 6 個月 27 天 / 成為病理專科醫師 2 年 1 個月 7 天
\end{itemize}

\hypertarget{ux672cux6587-content-18}{%
\subsection{本文 Content}\label{ux672cux6587-content-18}}

\begin{enumerate}
\def\labelenumi{\arabic{enumi}.}
\item
  罕見地失眠了

  因此整個早上腦袋都在亂放電,做的事情包含恍神,亂逛網路,對男友胡言亂語,看「明朝」{[}1{]},總之無法做正事。如果被不知道內情的人看到這個景象,一定會以為病理科醫師{[}2{]}都是薪水小偷。算了我也懶得澄清了。

  睡了個午覺以後整個狀態就好多了,可以完成一些日常雜事。但睡了午覺以後反而開始擔心,晚上又失眠怎麼辦?
\item
  不知道該不該寫的事情

  但還是寫好了。某個為了自己的喜好而安排院慶管弦樂團演出的院內高官,今天送了一條蜂蜜蛋糕給我,還很興高采烈地詢問我「如果院內這個管弦樂團,要辦全場音樂會的話可以嗎?」這當然意謂著一場報酬微薄,對病理工作毫無助益的大型編曲,但我還是答應了。一方面我自己喜歡出風頭,一方面人在屋簷下不得不低頭,不得不跪舔。

  好討厭沒骨氣又容易被虛名引誘的自己。
\item
  雜記:物價與其他{[}3{]}

  \begin{itemize}
  \item
    C院午餐72元(洋蔥豬柳,紫米飯,菜豆,苦瓜,白花菜),C院晚餐72元(洋蔥豬肉片,白米飯,芥蘭,高麗菜,白花菜),「光泉」頂級鮮奶優酪一杯40元(購於全家便利商店,很貴但值得,因為他真的徹底無糖且菌種很豐富,我就是吃這個跟低卡餐瘦下來的),「朝日」十六茶29元,「多喝水」香檳氣泡水26元(他之前出的櫻花跟玫瑰風味都超難喝,但這次終於成功了,那個淡淡的葡萄香非常宜人)
  \item
    機械輔助引體向上10次三組,機械胸推10次三組,二頭肌10次三組,三頭肌10次3組
  \end{itemize}
\end{enumerate}

\hypertarget{ux6ce8ux91cb-comment-18}{%
\subsection{注釋 Comment}\label{ux6ce8ux91cb-comment-18}}

{[}1{]} 指駱以軍的書。有關此書與其掀起的爭議,請見蒼白球日誌0043。

{[}2{]} 有關這個職業請見蒼白球日誌0011。

{[}3{]} 新台幣計價。有關新台幣請參見蒼白球日誌0007。

\hypertarget{ux9644ux9304-appendix-18}{%
\subsection{附錄 Appendix}\label{ux9644ux9304-appendix-18}}

\hypertarget{ux84bcux767dux7403ux65e5ux8a8c0053-20191120}{%
\section{蒼白球日誌0053
2019/11/20}\label{ux84bcux767dux7403ux65e5ux8a8c0053-20191120}}

\hypertarget{ux65e5ux671f-date-19}{%
\subsection{日期 Date}\label{ux65e5ux671f-date-19}}

\begin{itemize}
\tightlist
\item
  世界協調時間2019年(中華民國108年,令和1年)11月20日 / Unix 紀元 18220
  日 / 星期三 / 蒼白球紀元第53日
\item
  November 20, 2019 (UTC) / 18220 days since Unix Epoch / Wednesday /
  Globus Pallidum day 53
\item
  特殊註記:
\end{itemize}

\hypertarget{ux5e74ux9f61-age-19}{%
\subsection{年齡 Age}\label{ux5e74ux9f61-age-19}}

\begin{itemize}
\tightlist
\item
  33 years 6 months 28 days old / 2 years 1 months 8 days after
  acquiring ROC Surgical Pathology Licence
\item
  33 歲 6 個月 28 天 / 成為病理專科醫師 2 年 1 個月 8 天
\end{itemize}

\hypertarget{ux672cux6587-content-19}{%
\subsection{本文 Content}\label{ux672cux6587-content-19}}

\begin{enumerate}
\def\labelenumi{\arabic{enumi}.}
\item
  食記{[}1{]}

  這個月領優良醫師獎金10000元被秘書知道後,被秘書要求請客。於是今天與秘書一人,主治醫師兩人,住院醫師兩人,共六人一起去中友百貨的瓦城{[}2{]}用餐。

  由於制式的六人桌菜少了一些大家想吃的東西,所以又多叫了三道菜,共5972元,足足把我的獎金吃掉了快要六成,在正統的連鎖餐館大吃一頓可真不便宜啊!但,坦白講沒什麼好抱怨的,口味能夠讓大多數的台灣人大喊「好吃」,且品質穩定的桌菜不多,而瓦城剛好是其中一家,今天我們六個是真的開心地領教到什麼是好吃了,或許他貴也是有他的道理。

  然後,我們六個人居然把一大桌的菜吃到空空,兩個學弟真是好肚量。
\item
  欠了好多工作債

  聯篇歌曲「魔劍」拖了兩年還沒動工,下個月要申請的國科會計畫還沒寫,該做的研究還沒做,母后委託的清唱劇編曲還沒做,微積分還沒念,然後日常病理工作又不斷進逼。在多重壓力同時進逼的狀況下,或許唯一的方式只能繼續把魔劍往後拖延了吧,但我真的不甘心。

  我要一直記得魔劍,每天寫日記看到「成為病理專科醫師N年N月N天」的時候提醒自己,那就是魔劍起草以來的時間,而這作品已經等我這麼久了。
\item
  本來想要寫A院最近發生的變化

  以及這些變化導致的C院辦公室政治。但仔細想想,那個好像沒什麼特別的,很普通的醫院人事鬥爭而已,而且也暫時不影響我自己的大局。所以雖然是造成我很多心理負擔的事情,不過不值得描述。
\end{enumerate}

\hypertarget{ux6ce8ux91cb-comment-19}{%
\subsection{注釋 Comment}\label{ux6ce8ux91cb-comment-19}}

{[}1{]} 新台幣計價。有關新台幣請參見蒼白球日誌0007。

{[}2{]} 瓦城泰統股份有限公司(TTFB company
limited,英文縮寫TTFB,簡稱瓦城泰統集團),是一家臺灣的餐飲事業集團企業,總部位於臺灣新北市中和區,由徐承義等人創立於1990年。該企業旗下有瓦城泰國料理等知名於臺灣地區的餐飲連鎖店品牌。

\hypertarget{ux9644ux9304-appendix-19}{%
\subsection{附錄 Appendix}\label{ux9644ux9304-appendix-19}}

\hypertarget{ux84bcux767dux7403ux65e5ux8a8c0054-20191121}{%
\section{蒼白球日誌0054
2019/11/21}\label{ux84bcux767dux7403ux65e5ux8a8c0054-20191121}}

\hypertarget{ux65e5ux671f-date-20}{%
\subsection{日期 Date}\label{ux65e5ux671f-date-20}}

\begin{itemize}
\tightlist
\item
  世界協調時間2019年(中華民國108年,令和1年)11月21日 / Unix 紀元 18221
  日 / 星期四 / 蒼白球紀元第54日
\item
  November 21, 2019 (UTC) / 18221 days since Unix Epoch / Thursday /
  Globus Pallidum day 54
\item
  特殊註記:
\end{itemize}

\hypertarget{ux5e74ux9f61-age-20}{%
\subsection{年齡 Age}\label{ux5e74ux9f61-age-20}}

\begin{itemize}
\tightlist
\item
  33 years 6 months 29 days old / 2 years 1 months 9 days after
  acquiring ROC Surgical Pathology Licence
\item
  33 歲 6 個月 29 天 / 成為病理專科醫師 2 年 1 個月 9 天
\end{itemize}

\hypertarget{ux672cux6587-content-20}{%
\subsection{本文 Content}\label{ux672cux6587-content-20}}

\begin{enumerate}
\def\labelenumi{\arabic{enumi}.}
\item
  整個人都沒元氣

  今天整天昏昏欲睡,一直在辦公室打瞌睡,到底是因為昨天吃了太多糖分造成胰島素升高呢?還是因為前陣子累積的疲累呢?總之就算工作堆積如山,我今天還是得要早點睡,不然狀況永遠不會到達可以接受的範圍。
\item
  然後因為沒元氣所以教練課就\ldots\ldots{}

  炸掉了,完全不行,太久沒練又體力不對的狀況下,退回到九個月前的程度了。教練感覺很氣餒,可是我真的沒有辦法,病理正職跟研究步步進逼的時候,第一個犧牲的就是身體了。然後第二個犧牲的是音樂。金錢、知性跟外貌不能三個都要,而我老是直覺地先犧牲外貌事後再來後悔。可是,如果犧牲金錢成全外貌的話,也不一定就不會後悔吧。人間的缺陷滿地都是,無論朝哪一個方向前進都沒有一個滿意的解答。
\item
  雜記:物價與其他{[}1{]}

  \begin{itemize}
  \item
    C院午餐72元(紫米飯,排骨,高麗菜,菠菜,地瓜葉),C院晚餐72元(白米飯,雞腿,小白菜,茄子,山蘇),吃了一堆點心填肚子懶得記(今天不知道為什麼特別餓)
  \item
    「明朝」越看越好看
  \end{itemize}
\end{enumerate}

\hypertarget{ux6ce8ux91cb-comment-20}{%
\subsection{注釋 Comment}\label{ux6ce8ux91cb-comment-20}}

{[}1{]} 新台幣計價。有關新台幣請參見蒼白球日誌0007。

\hypertarget{ux9644ux9304-appendix-20}{%
\subsection{附錄 Appendix}\label{ux9644ux9304-appendix-20}}

\hypertarget{ux84bcux767dux7403ux65e5ux8a8c0055-20191122}{%
\section{蒼白球日誌0055
2019/11/22}\label{ux84bcux767dux7403ux65e5ux8a8c0055-20191122}}

\hypertarget{ux65e5ux671f-date-21}{%
\subsection{日期 Date}\label{ux65e5ux671f-date-21}}

\begin{itemize}
\tightlist
\item
  世界協調時間2019年(中華民國108年,令和1年)11月22日 / Unix 紀元 18222
  日 / 星期五 / 蒼白球紀元第55日
\item
  November 22, 2019 (UTC) / 18222 days since Unix Epoch / Friday /
  Globus Pallidum day 55
\item
  特殊註記:
\end{itemize}

\hypertarget{ux5e74ux9f61-age-21}{%
\subsection{年齡 Age}\label{ux5e74ux9f61-age-21}}

\begin{itemize}
\tightlist
\item
  33 years 6 months 30 days old / 2 years 1 months 10 days after
  acquiring ROC Surgical Pathology Licence
\item
  33 歲 6 個月 30 天 / 成為病理專科醫師 2 年 1 個月 10 天
\end{itemize}

\hypertarget{ux672cux6587-content-21}{%
\subsection{本文 Content}\label{ux672cux6587-content-21}}

\begin{enumerate}
\def\labelenumi{\arabic{enumi}.}
\item
  試印蒼白球日誌2019秋

  把這之前54天的日誌輸出成PDF檔之後,去C院附近的影印店試印了一本。成品意外地相當令人滿意,牛皮紙的顏色我很喜歡,膠裝的品質也在可接受範圍之內。

  而且202頁只收我144元{[}1{]}。(「印多本一點會算你更便宜喔!」)

  或許十二月初,蒼白球日誌2019秋正式截稿的時候,去影印店印就好了。

  成品附於附錄。
\item
  雜記:物價與其他

  \begin{itemize}
  \item
    H校今天因為老師有事所以停課。已經跟科內秘書先告假的我,於是用這個意外獲得的空閒時間處理了一些雜事,包含跟廠商洽談經費使用問題,之類的。系統提示:您的
    行政能力 上升了 2 點。
  \item
    C院午餐72元(紫米飯,獅子頭,高麗菜,青花菜,海帶),「清心福全」普洱23元(使用line點數兩點),「三商巧福」清燉牛肉麵115元,「光泉」豆漿牛奶32元
  \end{itemize}
\end{enumerate}

\hypertarget{ux6ce8ux91cb-comment-21}{%
\subsection{注釋 Comment}\label{ux6ce8ux91cb-comment-21}}

{[}1{]} 新台幣計價。有關新台幣請參見蒼白球日誌0007。

\hypertarget{ux9644ux9304-appendix-21}{%
\subsection{附錄 Appendix}\label{ux9644ux9304-appendix-21}}

蒼白球日誌2019秋試印成品 https://i.imgur.com/95xbg3W.jpg
https://i.imgur.com/efnzJDl.jpg

\hypertarget{ux84bcux767dux7403ux65e5ux8a8c0056-20191123}{%
\section{蒼白球日誌0056
2019/11/23}\label{ux84bcux767dux7403ux65e5ux8a8c0056-20191123}}

\hypertarget{ux65e5ux671f-date-22}{%
\subsection{日期 Date}\label{ux65e5ux671f-date-22}}

\begin{itemize}
\tightlist
\item
  世界協調時間2019年(中華民國108年,令和1年)11月23日 / Unix 紀元 18223
  日 / 星期六 / 蒼白球紀元第56日
\item
  November 23, 2019 (UTC) / 18223 days since Unix Epoch / Saturday /
  Globus Pallidum day 56
\item
  特殊註記:
\end{itemize}

\hypertarget{ux5e74ux9f61-age-22}{%
\subsection{年齡 Age}\label{ux5e74ux9f61-age-22}}

\begin{itemize}
\tightlist
\item
  33 years 7 months 0 days old / 2 years 1 months 11 days after
  acquiring ROC Surgical Pathology Licence
\item
  33 歲 7 個月 0 天 / 成為病理專科醫師 2 年 1 個月 11 天
\end{itemize}

\hypertarget{ux672cux6587-content-22}{%
\subsection{本文 Content}\label{ux672cux6587-content-22}}

\begin{enumerate}
\def\labelenumi{\arabic{enumi}.}
\item
  感冒了

  已經變成規律了,在高壓力生活期跟下一個高壓力生活期之間,那個短短的喘息時間就必定感冒,好像是免疫系統趁機在抗議那些工作的摧殘一樣。

  但沒辦法,我除了有好多事情要做以外,還得存錢。
\item
  母后大人今天一整天都在聯絡事情

  原因是我說真的撐不住了,沒有體力可以在期限內完成清唱劇的編曲,請他分一些工作出去。於是在經費尚稱足夠的狀態下,他把一些曲子發包給了別的編曲者。

  這樣的發包對我來說很重要,因為我為了變漂亮,得要健身並且好好睡覺。至於這樣會不會有點麻煩到母后,就再說吧,我不能為了成全這些工作而把自己想要的東西都捨棄。
\item
  統計學

  前陣子稍微看了點書,建立了對於微積分的初步概念,接下來得開始念統計了。發現wikibook的免費統計書寫得不錯,可以來念這個。
\end{enumerate}

\hypertarget{ux6ce8ux91cb-comment-22}{%
\subsection{注釋 Comment}\label{ux6ce8ux91cb-comment-22}}

\hypertarget{ux9644ux9304-appendix-22}{%
\subsection{附錄 Appendix}\label{ux9644ux9304-appendix-22}}

\hypertarget{ux84bcux767dux7403ux65e5ux8a8c0057-20191124}{%
\section{蒼白球日誌0057
2019/11/24}\label{ux84bcux767dux7403ux65e5ux8a8c0057-20191124}}

\hypertarget{ux65e5ux671f-date-23}{%
\subsection{日期 Date}\label{ux65e5ux671f-date-23}}

\begin{itemize}
\tightlist
\item
  世界協調時間2019年(中華民國108年,令和1年)11月24日 / Unix 紀元 18224
  日 / 星期日 / 蒼白球紀元第57日
\item
  November 24, 2019 (UTC) / 18224 days since Unix Epoch / Sunday /
  Globus Pallidum day 57
\end{itemize}

\hypertarget{ux5e74ux9f61-age-23}{%
\subsection{年齡 Age}\label{ux5e74ux9f61-age-23}}

\begin{itemize}
\tightlist
\item
  33 years 7 months 1 days old / 2 years 1 months 12 days after
  acquiring ROC Surgical Pathology Licence
\item
  33 歲 7 個月 1 天 / 成為病理專科醫師 2 年 1 個月 12 天
\end{itemize}

\hypertarget{ux672cux6587-content-23}{%
\subsection{本文 Content}\label{ux672cux6587-content-23}}

\begin{enumerate}
\def\labelenumi{\arabic{enumi}.}
\item
  學妹的第一次冰凍切片

  本來想要拜託我支援,但我在跟管樂團籌備清唱劇事宜,早上無法到醫院,只好說抱歉。於是學妹拜託學姐支援,兩位新科主治醫師協力,無奈的是依然想不出診斷,急得像熱鍋上的螞蟻,用facebook緊急連絡我。

  幸好這個診斷是在我可以隔空給建議的範圍之內,即時化解了一場恐慌。任何人的第一年主治醫師都是驚心動魄呢,我當年也是一樣。
\item
  本來想說今天一直在加班很累

  所以不想寫日誌,不過還是不要中斷好了,養成習慣。
\item
  雜記:物價與其他{[}1{]}

  中午請母后以及母后的生意夥伴吃「冒煙的喬」,2066元。優良醫師獎金一萬塊,結果請客請掉了八千多\ldots\ldots 這種事情被知道實在是很虧啊。
\end{enumerate}

\hypertarget{ux6ce8ux91cb-comment-23}{%
\subsection{注釋 Comment}\label{ux6ce8ux91cb-comment-23}}

{[}1{]} 新台幣計價。有關新台幣請參見蒼白球日誌0007。

\hypertarget{ux9644ux9304-appendix-23}{%
\subsection{附錄 Appendix}\label{ux9644ux9304-appendix-23}}

\hypertarget{ux84bcux767dux7403ux65e5ux8a8c0058-20191125}{%
\section{蒼白球日誌0058
2019/11/25}\label{ux84bcux767dux7403ux65e5ux8a8c0058-20191125}}

\hypertarget{ux65e5ux671f-date-24}{%
\subsection{日期 Date}\label{ux65e5ux671f-date-24}}

\begin{itemize}
\tightlist
\item
  世界協調時間2019年(中華民國108年,令和1年)11月25日 / Unix 紀元 18225
  日 / 星期一 / 蒼白球紀元第58日
\item
  November 25, 2019 (UTC) / 18225 days since Unix Epoch / Monday /
  Globus Pallidum day 58
\item
  特殊註記:
\end{itemize}

\hypertarget{ux5e74ux9f61-age-24}{%
\subsection{年齡 Age}\label{ux5e74ux9f61-age-24}}

\begin{itemize}
\tightlist
\item
  33 years 7 months 2 days old / 2 years 1 months 13 days after
  acquiring ROC Surgical Pathology Licence
\item
  33 歲 7 個月 2 天 / 成為病理專科醫師 2 年 1 個月 13 天
\end{itemize}

\hypertarget{ux672cux6587-content-24}{%
\subsection{本文 Content}\label{ux672cux6587-content-24}}

\begin{enumerate}
\def\labelenumi{\arabic{enumi}.}
\item
  清唱劇的進度非常緊迫

  所以今天趕快找人調班,這樣我在十二月初才能獲得三天假完成所有編曲。

  結果被某學長拒絕了,逼不得已只好找主任,主任很爽朗地答應了。常常找主管支援真的挺不好意思的。
\item
  找主任換班的時候他說

  「好啊沒問題這個班可以換」

  「突然想到,蒼白球我好像有事找你,但是\ldots..ㄟ\ldots..想不出來是什麼事情,所以\ldots\ldots」

  然後就中斷了,也太健忘了吧!不過如果是這種會輕易忘記的,我想應該不會是什麼嚴重的事情才對。暫時安心?
\item
  片子滿出來了

  堆積如山,所以沒空好好寫字,日誌只好隨便寫了。
\end{enumerate}

\hypertarget{ux6ce8ux91cb-comment-24}{%
\subsection{注釋 Comment}\label{ux6ce8ux91cb-comment-24}}

\hypertarget{ux9644ux9304-appendix-24}{%
\subsection{附錄 Appendix}\label{ux9644ux9304-appendix-24}}

\hypertarget{ux84bcux767dux7403ux65e5ux8a8c0059-20191126}{%
\section{蒼白球日誌0059
2019/11/26}\label{ux84bcux767dux7403ux65e5ux8a8c0059-20191126}}

\hypertarget{ux65e5ux671f-date-25}{%
\subsection{日期 Date}\label{ux65e5ux671f-date-25}}

\begin{itemize}
\tightlist
\item
  世界協調時間2019年(中華民國108年,令和1年)11月26日 / Unix 紀元 18226
  日 / 星期二 / 蒼白球紀元第59日
\item
  November 26, 2019 (UTC) / 18226 days since Unix Epoch / Tuesday /
  Globus Pallidum day 59
\item
  特殊註記:
\end{itemize}

\hypertarget{ux5e74ux9f61-age-25}{%
\subsection{年齡 Age}\label{ux5e74ux9f61-age-25}}

\begin{itemize}
\tightlist
\item
  33 years 7 months 3 days old / 2 years 1 months 14 days after
  acquiring ROC Surgical Pathology Licence
\item
  33 歲 7 個月 3 天 / 成為病理專科醫師 2 年 1 個月 14 天
\end{itemize}

\hypertarget{ux672cux6587-content-25}{%
\subsection{本文 Content}\label{ux672cux6587-content-25}}

\begin{enumerate}
\def\labelenumi{\arabic{enumi}.}
\item
  有關一中益民商圈(今天去逛街順便寫寫)

  在台中一中正北方,占地廣闊的這個繁華商店街,五層樓的大型社區式建築,興建於2010年。由於一二樓的各種小吃以及服飾店非常引人注目,逛街時很難去特別留意三四五樓蓋了什麼東西。

  後來因為某些原因(\textbf{點到為止不解釋,張老師有云,知道的人就知道,不知道的人不知道也無所謂})必須造訪各種陌生人的住處,才知道原來益民商圈的三四五樓全部都隔成四到八坪的套房,裝潢頗素雅,且設施齊全。地下室則有台中市少見的大型停車場。

  用眼睛估計的話,在這個空間裡面,若沒有一千間套房,至少也有五六百間,頗多年輕人,或者不是年輕只是很魯的人因為良好的舒適度跟相對合理的房租而寄居於此。似乎只要有正當工作,人就可以躲藏在益民子宮裡面,以生活機能為臍帶,躲藏到月月年年,不用與一中以外的世界接觸,儼然是一個國中之國。

  或許台中一中真的是一個大型的胎盤吧。有些人寄居在一中國,直接在物理上由一中母體供應生活,並沒有居住在一中國的我們,臍帶卻也斷不掉,持續被那個母體影響著所有的潛意識。
\item
  食記{[}1{]}

  益民居然有50元的香腸炒飯,猜想應該是因為房子自有,所以沒有因為房地產炒作而影響到租金成本。至於50元炒飯的味道,痾,就\ldots\ldots 人家已經賣這麼便宜了,也不好意思要求什麼。
\end{enumerate}

\hypertarget{ux6ce8ux91cb-comment-25}{%
\subsection{注釋 Comment}\label{ux6ce8ux91cb-comment-25}}

{[}1{]} 新台幣計價。有關新台幣請參見蒼白球日誌0007。

\hypertarget{ux9644ux9304-appendix-25}{%
\subsection{附錄 Appendix}\label{ux9644ux9304-appendix-25}}

\hypertarget{ux84bcux767dux7403ux65e5ux8a8c0060-20191127}{%
\section{蒼白球日誌0060
2019/11/27}\label{ux84bcux767dux7403ux65e5ux8a8c0060-20191127}}

\hypertarget{ux65e5ux671f-date-26}{%
\subsection{日期 Date}\label{ux65e5ux671f-date-26}}

\begin{itemize}
\tightlist
\item
  世界協調時間2019年(中華民國108年,令和1年)11月27日 / Unix 紀元 18227
  日 / 星期三 / 蒼白球紀元第60日
\item
  November 27, 2019 (UTC) / 18227 days since Unix Epoch / Wednesday /
  Globus Pallidum day 60
\item
  特殊註記:
\end{itemize}

\hypertarget{ux5e74ux9f61-age-26}{%
\subsection{年齡 Age}\label{ux5e74ux9f61-age-26}}

\begin{itemize}
\tightlist
\item
  33 years 7 months 4 days old / 2 years 1 months 15 days after
  acquiring ROC Surgical Pathology Licence
\item
  33 歲 7 個月 4 天 / 成為病理專科醫師 2 年 1 個月 15 天
\end{itemize}

\hypertarget{ux672cux6587-content-26}{%
\subsection{本文 Content}\label{ux672cux6587-content-26}}

工作堆積如山又重感冒,覺得再這樣下去真的要病倒了的時候,突然遇到救星:C院鄰近突然開了一家賣各種雞湯的店,包含九尾草雞、山葡萄雞、四物雞等等懷念的湯品。一份200元{[}1{]}實在是貴,但在急需雞湯的狀態下我只好付錢了。

吃完一份九尾草雞以後覺得好多了,應該可以多做一點工作了。但我很擔心這家價位偏高,賣得東西又不是日常吃食的店能開多久。或許三個月吧。

\hypertarget{ux6ce8ux91cb-comment-26}{%
\subsection{注釋 Comment}\label{ux6ce8ux91cb-comment-26}}

{[}1{]} 新台幣計價。有關新台幣請參見蒼白球日誌0007。

\hypertarget{ux9644ux9304-appendix-26}{%
\subsection{附錄 Appendix}\label{ux9644ux9304-appendix-26}}

\hypertarget{ux84bcux767dux7403ux65e5ux8a8c0061-20191128}{%
\section{蒼白球日誌0061
2019/11/28}\label{ux84bcux767dux7403ux65e5ux8a8c0061-20191128}}

\hypertarget{ux65e5ux671f-date-27}{%
\subsection{日期 Date}\label{ux65e5ux671f-date-27}}

\begin{itemize}
\tightlist
\item
  世界協調時間2019年(中華民國108年,令和1年)11月28日 / Unix 紀元 18228
  日 / 星期四 / 蒼白球紀元第61日
\item
  November 28, 2019 (UTC) / 18228 days since Unix Epoch / Thursday /
  Globus Pallidum day 61
\item
  特殊註記:
\end{itemize}

\hypertarget{ux5e74ux9f61-age-27}{%
\subsection{年齡 Age}\label{ux5e74ux9f61-age-27}}

\begin{itemize}
\tightlist
\item
  33 years 7 months 5 days old / 2 years 1 months 16 days after
  acquiring ROC Surgical Pathology Licence
\item
  33 歲 7 個月 5 天 / 成為病理專科醫師 2 年 1 個月 16 天
\end{itemize}

\hypertarget{ux672cux6587-content-27}{%
\subsection{本文 Content}\label{ux672cux6587-content-27}}

電腦版的facebook壞掉了,而且重感冒,只能隨便這樣寫一行代表沒有斷更

\hypertarget{ux6ce8ux91cb-comment-27}{%
\subsection{注釋 Comment}\label{ux6ce8ux91cb-comment-27}}

\hypertarget{ux9644ux9304-appendix-27}{%
\subsection{附錄 Appendix}\label{ux9644ux9304-appendix-27}}

\hypertarget{ux84bcux767dux7403ux65e5ux8a8c0062-20191129}{%
\section{蒼白球日誌0062
2019/11/29}\label{ux84bcux767dux7403ux65e5ux8a8c0062-20191129}}

\hypertarget{ux65e5ux671f-date-28}{%
\subsection{日期 Date}\label{ux65e5ux671f-date-28}}

\begin{itemize}
\tightlist
\item
  世界協調時間2019年(中華民國108年,令和1年)11月29日 / Unix 紀元 18229
  日 / 星期五 / 蒼白球紀元第62日
\item
  November 29, 2019 (UTC) / 18229 days since Unix Epoch / Friday /
  Globus Pallidum day 62
\item
  特殊註記:
\end{itemize}

\hypertarget{ux5e74ux9f61-age-28}{%
\subsection{年齡 Age}\label{ux5e74ux9f61-age-28}}

\begin{itemize}
\tightlist
\item
  33 years 7 months 6 days old / 2 years 1 months 17 days after
  acquiring ROC Surgical Pathology Licence
\item
  33 歲 7 個月 6 天 / 成為病理專科醫師 2 年 1 個月 17 天
\end{itemize}

\hypertarget{ux672cux6587-content-28}{%
\subsection{本文 Content}\label{ux672cux6587-content-28}}

\begin{enumerate}
\def\labelenumi{\arabic{enumi}.}
\item
  有機合成這檔事

  今天H大的課是蛋白質結構,講蛋白質結構免不了要稍微提到藥物合成,於是老師就拿出了一個結構式,說明小分子藥物在對結構做各種變化的時候,藥效會因為與蛋白質的互動不同有什麼樣的變化。對於這件事情我不免疑惑,忍不住問老師:「這種合成可以想做什麼就做什麼嗎?」

  老師立刻回答說:「可以,用一些保護基加一點設計就可以。詳情請見有機化學課本,並且不要質疑做有機合成的人的威能。」{[}1{]}

  原來有機合成真的已經進化到近乎隨心所欲的程度了,雖然已經把化學課本忘光光,但是聽到這種幾乎是魔法的神奇科技,還是馬上讓小時候的化學魂有點燃燒了起來,我的心在跟我說「好想知道那個怎麼做啊!好想知道魔法啊!」。

  然而我已經太老了,即使想學習這個知識也已經沒有時間跟管道,只能無奈地接受人家在變魔法,而我只能看痔瘡的現實。
\item
  雜記:物價與其他{[}2{]}

  \begin{itemize}
  \item
    肉羹麵55元,「清心」普洱4元(使用line點數21點),「丸龜製麵」大碗牛肉烏龍麵164元,「花茶大師」涼爽退火茶50元(就是仙草薄荷茶)
  \item
    上教練課好痛苦,教練(基於職業習慣)硬是問我工作的事情然後問了又不太相信好尷尬(編曲看痔瘡讀博士這種話兜在一起就很像騙人沒錯),進步好慢(不怪教練因為我真的大部分時間都累到沒運動),我到底哪一天才能變漂亮
  \end{itemize}
\end{enumerate}

\hypertarget{ux6ce8ux91cb-comment-28}{%
\subsection{注釋 Comment}\label{ux6ce8ux91cb-comment-28}}

{[}1{]} 這個老師合作的有機合成家是FJM老師,原來FJM這麼威啊\ldots\ldots{}

{[}2{]} 新台幣計價。有關新台幣請參見蒼白球日誌0007。

\hypertarget{ux9644ux9304-appendix-28}{%
\subsection{附錄 Appendix}\label{ux9644ux9304-appendix-28}}

\hypertarget{ux84bcux767dux7403ux65e5ux8a8c0063-20191130}{%
\section{蒼白球日誌0063
2019/11/30}\label{ux84bcux767dux7403ux65e5ux8a8c0063-20191130}}

\hypertarget{ux65e5ux671f-date-29}{%
\subsection{日期 Date}\label{ux65e5ux671f-date-29}}

\begin{itemize}
\tightlist
\item
  世界協調時間2019年(中華民國108年,令和1年)11月30日 / Unix 紀元 18230
  日 / 星期六 / 蒼白球紀元第63日
\item
  November 30, 2019 (UTC) / 18230 days since Unix Epoch / Saturday /
  Globus Pallidus Record day 63
\item
  特殊註記:
\end{itemize}

\hypertarget{ux5e74ux9f61-age-29}{%
\subsection{年齡 Age}\label{ux5e74ux9f61-age-29}}

\begin{itemize}
\tightlist
\item
  33 years 7 months 7 days old / 2 years 1 months 18 days after
  acquiring ROC Surgical Pathology Licence
\item
  33 歲 7 個月 7 天 / 成為病理專科醫師 2 年 1 個月 18 天
\end{itemize}

\hypertarget{ux672cux6587-content-29}{%
\subsection{本文 Content}\label{ux672cux6587-content-29}}

今天稍微放下工作,放空了一下,看著蒼白球日誌2019秋前半段的試印本,回憶這幾個月發生的事情,然後把它丟到垃圾桶。

因為在這篇寫完以後,我就可以開始排版2019秋的正式版本了,所以不再需要那個試印本。

至於這個丟進垃圾桶的舉動,會不會造成蒼白球日誌落入不知名人士手上的風險?坦白講我覺得被撿到更好,等於成功在這島上的某處製造一個備份。

\hypertarget{ux6ce8ux91cb-comment-29}{%
\subsection{注釋 Comment}\label{ux6ce8ux91cb-comment-29}}

\hypertarget{ux9644ux9304-appendix-29}{%
\subsection{附錄 Appendix}\label{ux9644ux9304-appendix-29}}

\end{document}
